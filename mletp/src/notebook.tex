
% Default to the notebook output style

    


% Inherit from the specified cell style.




    
\documentclass[11pt]{article}

    
    
    \usepackage[T1]{fontenc}
    % Nicer default font (+ math font) than Computer Modern for most use cases
    \usepackage{mathpazo}

    % Basic figure setup, for now with no caption control since it's done
    % automatically by Pandoc (which extracts ![](path) syntax from Markdown).
    \usepackage{graphicx}
    % We will generate all images so they have a width \maxwidth. This means
    % that they will get their normal width if they fit onto the page, but
    % are scaled down if they would overflow the margins.
    \makeatletter
    \def\maxwidth{\ifdim\Gin@nat@width>\linewidth\linewidth
    \else\Gin@nat@width\fi}
    \makeatother
    \let\Oldincludegraphics\includegraphics
    % Set max figure width to be 80% of text width, for now hardcoded.
    \renewcommand{\includegraphics}[1]{\Oldincludegraphics[width=.8\maxwidth]{#1}}
    % Ensure that by default, figures have no caption (until we provide a
    % proper Figure object with a Caption API and a way to capture that
    % in the conversion process - todo).
    \usepackage{caption}
    \DeclareCaptionLabelFormat{nolabel}{}
    \captionsetup{labelformat=nolabel}

    \usepackage{adjustbox} % Used to constrain images to a maximum size 
    \usepackage{xcolor} % Allow colors to be defined
    \usepackage{enumerate} % Needed for markdown enumerations to work
    \usepackage{geometry} % Used to adjust the document margins
    \usepackage{amsmath} % Equations
    \usepackage{amssymb} % Equations
    \usepackage{textcomp} % defines textquotesingle
    % Hack from http://tex.stackexchange.com/a/47451/13684:
    \AtBeginDocument{%
        \def\PYZsq{\textquotesingle}% Upright quotes in Pygmentized code
    }
    \usepackage{upquote} % Upright quotes for verbatim code
    \usepackage{eurosym} % defines \euro
    \usepackage[mathletters]{ucs} % Extended unicode (utf-8) support
    \usepackage[utf8x]{inputenc} % Allow utf-8 characters in the tex document
    \usepackage{fancyvrb} % verbatim replacement that allows latex
    \usepackage{grffile} % extends the file name processing of package graphics 
                         % to support a larger range 
    % The hyperref package gives us a pdf with properly built
    % internal navigation ('pdf bookmarks' for the table of contents,
    % internal cross-reference links, web links for URLs, etc.)
    \usepackage{hyperref}
    \usepackage{longtable} % longtable support required by pandoc >1.10
    \usepackage{booktabs}  % table support for pandoc > 1.12.2
    \usepackage[inline]{enumitem} % IRkernel/repr support (it uses the enumerate* environment)
    \usepackage[normalem]{ulem} % ulem is needed to support strikethroughs (\sout)
                                % normalem makes italics be italics, not underlines
    

    
    
    % Colors for the hyperref package
    \definecolor{urlcolor}{rgb}{0,.145,.698}
    \definecolor{linkcolor}{rgb}{.71,0.21,0.01}
    \definecolor{citecolor}{rgb}{.12,.54,.11}

    % ANSI colors
    \definecolor{ansi-black}{HTML}{3E424D}
    \definecolor{ansi-black-intense}{HTML}{282C36}
    \definecolor{ansi-red}{HTML}{E75C58}
    \definecolor{ansi-red-intense}{HTML}{B22B31}
    \definecolor{ansi-green}{HTML}{00A250}
    \definecolor{ansi-green-intense}{HTML}{007427}
    \definecolor{ansi-yellow}{HTML}{DDB62B}
    \definecolor{ansi-yellow-intense}{HTML}{B27D12}
    \definecolor{ansi-blue}{HTML}{208FFB}
    \definecolor{ansi-blue-intense}{HTML}{0065CA}
    \definecolor{ansi-magenta}{HTML}{D160C4}
    \definecolor{ansi-magenta-intense}{HTML}{A03196}
    \definecolor{ansi-cyan}{HTML}{60C6C8}
    \definecolor{ansi-cyan-intense}{HTML}{258F8F}
    \definecolor{ansi-white}{HTML}{C5C1B4}
    \definecolor{ansi-white-intense}{HTML}{A1A6B2}

    % commands and environments needed by pandoc snippets
    % extracted from the output of `pandoc -s`
    \providecommand{\tightlist}{%
      \setlength{\itemsep}{0pt}\setlength{\parskip}{0pt}}
    \DefineVerbatimEnvironment{Highlighting}{Verbatim}{commandchars=\\\{\}}
    % Add ',fontsize=\small' for more characters per line
    \newenvironment{Shaded}{}{}
    \newcommand{\KeywordTok}[1]{\textcolor[rgb]{0.00,0.44,0.13}{\textbf{{#1}}}}
    \newcommand{\DataTypeTok}[1]{\textcolor[rgb]{0.56,0.13,0.00}{{#1}}}
    \newcommand{\DecValTok}[1]{\textcolor[rgb]{0.25,0.63,0.44}{{#1}}}
    \newcommand{\BaseNTok}[1]{\textcolor[rgb]{0.25,0.63,0.44}{{#1}}}
    \newcommand{\FloatTok}[1]{\textcolor[rgb]{0.25,0.63,0.44}{{#1}}}
    \newcommand{\CharTok}[1]{\textcolor[rgb]{0.25,0.44,0.63}{{#1}}}
    \newcommand{\StringTok}[1]{\textcolor[rgb]{0.25,0.44,0.63}{{#1}}}
    \newcommand{\CommentTok}[1]{\textcolor[rgb]{0.38,0.63,0.69}{\textit{{#1}}}}
    \newcommand{\OtherTok}[1]{\textcolor[rgb]{0.00,0.44,0.13}{{#1}}}
    \newcommand{\AlertTok}[1]{\textcolor[rgb]{1.00,0.00,0.00}{\textbf{{#1}}}}
    \newcommand{\FunctionTok}[1]{\textcolor[rgb]{0.02,0.16,0.49}{{#1}}}
    \newcommand{\RegionMarkerTok}[1]{{#1}}
    \newcommand{\ErrorTok}[1]{\textcolor[rgb]{1.00,0.00,0.00}{\textbf{{#1}}}}
    \newcommand{\NormalTok}[1]{{#1}}
    
    % Additional commands for more recent versions of Pandoc
    \newcommand{\ConstantTok}[1]{\textcolor[rgb]{0.53,0.00,0.00}{{#1}}}
    \newcommand{\SpecialCharTok}[1]{\textcolor[rgb]{0.25,0.44,0.63}{{#1}}}
    \newcommand{\VerbatimStringTok}[1]{\textcolor[rgb]{0.25,0.44,0.63}{{#1}}}
    \newcommand{\SpecialStringTok}[1]{\textcolor[rgb]{0.73,0.40,0.53}{{#1}}}
    \newcommand{\ImportTok}[1]{{#1}}
    \newcommand{\DocumentationTok}[1]{\textcolor[rgb]{0.73,0.13,0.13}{\textit{{#1}}}}
    \newcommand{\AnnotationTok}[1]{\textcolor[rgb]{0.38,0.63,0.69}{\textbf{\textit{{#1}}}}}
    \newcommand{\CommentVarTok}[1]{\textcolor[rgb]{0.38,0.63,0.69}{\textbf{\textit{{#1}}}}}
    \newcommand{\VariableTok}[1]{\textcolor[rgb]{0.10,0.09,0.49}{{#1}}}
    \newcommand{\ControlFlowTok}[1]{\textcolor[rgb]{0.00,0.44,0.13}{\textbf{{#1}}}}
    \newcommand{\OperatorTok}[1]{\textcolor[rgb]{0.40,0.40,0.40}{{#1}}}
    \newcommand{\BuiltInTok}[1]{{#1}}
    \newcommand{\ExtensionTok}[1]{{#1}}
    \newcommand{\PreprocessorTok}[1]{\textcolor[rgb]{0.74,0.48,0.00}{{#1}}}
    \newcommand{\AttributeTok}[1]{\textcolor[rgb]{0.49,0.56,0.16}{{#1}}}
    \newcommand{\InformationTok}[1]{\textcolor[rgb]{0.38,0.63,0.69}{\textbf{\textit{{#1}}}}}
    \newcommand{\WarningTok}[1]{\textcolor[rgb]{0.38,0.63,0.69}{\textbf{\textit{{#1}}}}}
    
    
    % Define a nice break command that doesn't care if a line doesn't already
    % exist.
    \def\br{\hspace*{\fill} \\* }
    % Math Jax compatability definitions
    \def\gt{>}
    \def\lt{<}
    % Document parameters
    \title{MachineLearningCode}
    
    
    

    % Pygments definitions
    
\makeatletter
\def\PY@reset{\let\PY@it=\relax \let\PY@bf=\relax%
    \let\PY@ul=\relax \let\PY@tc=\relax%
    \let\PY@bc=\relax \let\PY@ff=\relax}
\def\PY@tok#1{\csname PY@tok@#1\endcsname}
\def\PY@toks#1+{\ifx\relax#1\empty\else%
    \PY@tok{#1}\expandafter\PY@toks\fi}
\def\PY@do#1{\PY@bc{\PY@tc{\PY@ul{%
    \PY@it{\PY@bf{\PY@ff{#1}}}}}}}
\def\PY#1#2{\PY@reset\PY@toks#1+\relax+\PY@do{#2}}

\expandafter\def\csname PY@tok@w\endcsname{\def\PY@tc##1{\textcolor[rgb]{0.73,0.73,0.73}{##1}}}
\expandafter\def\csname PY@tok@c\endcsname{\let\PY@it=\textit\def\PY@tc##1{\textcolor[rgb]{0.25,0.50,0.50}{##1}}}
\expandafter\def\csname PY@tok@cp\endcsname{\def\PY@tc##1{\textcolor[rgb]{0.74,0.48,0.00}{##1}}}
\expandafter\def\csname PY@tok@k\endcsname{\let\PY@bf=\textbf\def\PY@tc##1{\textcolor[rgb]{0.00,0.50,0.00}{##1}}}
\expandafter\def\csname PY@tok@kp\endcsname{\def\PY@tc##1{\textcolor[rgb]{0.00,0.50,0.00}{##1}}}
\expandafter\def\csname PY@tok@kt\endcsname{\def\PY@tc##1{\textcolor[rgb]{0.69,0.00,0.25}{##1}}}
\expandafter\def\csname PY@tok@o\endcsname{\def\PY@tc##1{\textcolor[rgb]{0.40,0.40,0.40}{##1}}}
\expandafter\def\csname PY@tok@ow\endcsname{\let\PY@bf=\textbf\def\PY@tc##1{\textcolor[rgb]{0.67,0.13,1.00}{##1}}}
\expandafter\def\csname PY@tok@nb\endcsname{\def\PY@tc##1{\textcolor[rgb]{0.00,0.50,0.00}{##1}}}
\expandafter\def\csname PY@tok@nf\endcsname{\def\PY@tc##1{\textcolor[rgb]{0.00,0.00,1.00}{##1}}}
\expandafter\def\csname PY@tok@nc\endcsname{\let\PY@bf=\textbf\def\PY@tc##1{\textcolor[rgb]{0.00,0.00,1.00}{##1}}}
\expandafter\def\csname PY@tok@nn\endcsname{\let\PY@bf=\textbf\def\PY@tc##1{\textcolor[rgb]{0.00,0.00,1.00}{##1}}}
\expandafter\def\csname PY@tok@ne\endcsname{\let\PY@bf=\textbf\def\PY@tc##1{\textcolor[rgb]{0.82,0.25,0.23}{##1}}}
\expandafter\def\csname PY@tok@nv\endcsname{\def\PY@tc##1{\textcolor[rgb]{0.10,0.09,0.49}{##1}}}
\expandafter\def\csname PY@tok@no\endcsname{\def\PY@tc##1{\textcolor[rgb]{0.53,0.00,0.00}{##1}}}
\expandafter\def\csname PY@tok@nl\endcsname{\def\PY@tc##1{\textcolor[rgb]{0.63,0.63,0.00}{##1}}}
\expandafter\def\csname PY@tok@ni\endcsname{\let\PY@bf=\textbf\def\PY@tc##1{\textcolor[rgb]{0.60,0.60,0.60}{##1}}}
\expandafter\def\csname PY@tok@na\endcsname{\def\PY@tc##1{\textcolor[rgb]{0.49,0.56,0.16}{##1}}}
\expandafter\def\csname PY@tok@nt\endcsname{\let\PY@bf=\textbf\def\PY@tc##1{\textcolor[rgb]{0.00,0.50,0.00}{##1}}}
\expandafter\def\csname PY@tok@nd\endcsname{\def\PY@tc##1{\textcolor[rgb]{0.67,0.13,1.00}{##1}}}
\expandafter\def\csname PY@tok@s\endcsname{\def\PY@tc##1{\textcolor[rgb]{0.73,0.13,0.13}{##1}}}
\expandafter\def\csname PY@tok@sd\endcsname{\let\PY@it=\textit\def\PY@tc##1{\textcolor[rgb]{0.73,0.13,0.13}{##1}}}
\expandafter\def\csname PY@tok@si\endcsname{\let\PY@bf=\textbf\def\PY@tc##1{\textcolor[rgb]{0.73,0.40,0.53}{##1}}}
\expandafter\def\csname PY@tok@se\endcsname{\let\PY@bf=\textbf\def\PY@tc##1{\textcolor[rgb]{0.73,0.40,0.13}{##1}}}
\expandafter\def\csname PY@tok@sr\endcsname{\def\PY@tc##1{\textcolor[rgb]{0.73,0.40,0.53}{##1}}}
\expandafter\def\csname PY@tok@ss\endcsname{\def\PY@tc##1{\textcolor[rgb]{0.10,0.09,0.49}{##1}}}
\expandafter\def\csname PY@tok@sx\endcsname{\def\PY@tc##1{\textcolor[rgb]{0.00,0.50,0.00}{##1}}}
\expandafter\def\csname PY@tok@m\endcsname{\def\PY@tc##1{\textcolor[rgb]{0.40,0.40,0.40}{##1}}}
\expandafter\def\csname PY@tok@gh\endcsname{\let\PY@bf=\textbf\def\PY@tc##1{\textcolor[rgb]{0.00,0.00,0.50}{##1}}}
\expandafter\def\csname PY@tok@gu\endcsname{\let\PY@bf=\textbf\def\PY@tc##1{\textcolor[rgb]{0.50,0.00,0.50}{##1}}}
\expandafter\def\csname PY@tok@gd\endcsname{\def\PY@tc##1{\textcolor[rgb]{0.63,0.00,0.00}{##1}}}
\expandafter\def\csname PY@tok@gi\endcsname{\def\PY@tc##1{\textcolor[rgb]{0.00,0.63,0.00}{##1}}}
\expandafter\def\csname PY@tok@gr\endcsname{\def\PY@tc##1{\textcolor[rgb]{1.00,0.00,0.00}{##1}}}
\expandafter\def\csname PY@tok@ge\endcsname{\let\PY@it=\textit}
\expandafter\def\csname PY@tok@gs\endcsname{\let\PY@bf=\textbf}
\expandafter\def\csname PY@tok@gp\endcsname{\let\PY@bf=\textbf\def\PY@tc##1{\textcolor[rgb]{0.00,0.00,0.50}{##1}}}
\expandafter\def\csname PY@tok@go\endcsname{\def\PY@tc##1{\textcolor[rgb]{0.53,0.53,0.53}{##1}}}
\expandafter\def\csname PY@tok@gt\endcsname{\def\PY@tc##1{\textcolor[rgb]{0.00,0.27,0.87}{##1}}}
\expandafter\def\csname PY@tok@err\endcsname{\def\PY@bc##1{\setlength{\fboxsep}{0pt}\fcolorbox[rgb]{1.00,0.00,0.00}{1,1,1}{\strut ##1}}}
\expandafter\def\csname PY@tok@kc\endcsname{\let\PY@bf=\textbf\def\PY@tc##1{\textcolor[rgb]{0.00,0.50,0.00}{##1}}}
\expandafter\def\csname PY@tok@kd\endcsname{\let\PY@bf=\textbf\def\PY@tc##1{\textcolor[rgb]{0.00,0.50,0.00}{##1}}}
\expandafter\def\csname PY@tok@kn\endcsname{\let\PY@bf=\textbf\def\PY@tc##1{\textcolor[rgb]{0.00,0.50,0.00}{##1}}}
\expandafter\def\csname PY@tok@kr\endcsname{\let\PY@bf=\textbf\def\PY@tc##1{\textcolor[rgb]{0.00,0.50,0.00}{##1}}}
\expandafter\def\csname PY@tok@bp\endcsname{\def\PY@tc##1{\textcolor[rgb]{0.00,0.50,0.00}{##1}}}
\expandafter\def\csname PY@tok@fm\endcsname{\def\PY@tc##1{\textcolor[rgb]{0.00,0.00,1.00}{##1}}}
\expandafter\def\csname PY@tok@vc\endcsname{\def\PY@tc##1{\textcolor[rgb]{0.10,0.09,0.49}{##1}}}
\expandafter\def\csname PY@tok@vg\endcsname{\def\PY@tc##1{\textcolor[rgb]{0.10,0.09,0.49}{##1}}}
\expandafter\def\csname PY@tok@vi\endcsname{\def\PY@tc##1{\textcolor[rgb]{0.10,0.09,0.49}{##1}}}
\expandafter\def\csname PY@tok@vm\endcsname{\def\PY@tc##1{\textcolor[rgb]{0.10,0.09,0.49}{##1}}}
\expandafter\def\csname PY@tok@sa\endcsname{\def\PY@tc##1{\textcolor[rgb]{0.73,0.13,0.13}{##1}}}
\expandafter\def\csname PY@tok@sb\endcsname{\def\PY@tc##1{\textcolor[rgb]{0.73,0.13,0.13}{##1}}}
\expandafter\def\csname PY@tok@sc\endcsname{\def\PY@tc##1{\textcolor[rgb]{0.73,0.13,0.13}{##1}}}
\expandafter\def\csname PY@tok@dl\endcsname{\def\PY@tc##1{\textcolor[rgb]{0.73,0.13,0.13}{##1}}}
\expandafter\def\csname PY@tok@s2\endcsname{\def\PY@tc##1{\textcolor[rgb]{0.73,0.13,0.13}{##1}}}
\expandafter\def\csname PY@tok@sh\endcsname{\def\PY@tc##1{\textcolor[rgb]{0.73,0.13,0.13}{##1}}}
\expandafter\def\csname PY@tok@s1\endcsname{\def\PY@tc##1{\textcolor[rgb]{0.73,0.13,0.13}{##1}}}
\expandafter\def\csname PY@tok@mb\endcsname{\def\PY@tc##1{\textcolor[rgb]{0.40,0.40,0.40}{##1}}}
\expandafter\def\csname PY@tok@mf\endcsname{\def\PY@tc##1{\textcolor[rgb]{0.40,0.40,0.40}{##1}}}
\expandafter\def\csname PY@tok@mh\endcsname{\def\PY@tc##1{\textcolor[rgb]{0.40,0.40,0.40}{##1}}}
\expandafter\def\csname PY@tok@mi\endcsname{\def\PY@tc##1{\textcolor[rgb]{0.40,0.40,0.40}{##1}}}
\expandafter\def\csname PY@tok@il\endcsname{\def\PY@tc##1{\textcolor[rgb]{0.40,0.40,0.40}{##1}}}
\expandafter\def\csname PY@tok@mo\endcsname{\def\PY@tc##1{\textcolor[rgb]{0.40,0.40,0.40}{##1}}}
\expandafter\def\csname PY@tok@ch\endcsname{\let\PY@it=\textit\def\PY@tc##1{\textcolor[rgb]{0.25,0.50,0.50}{##1}}}
\expandafter\def\csname PY@tok@cm\endcsname{\let\PY@it=\textit\def\PY@tc##1{\textcolor[rgb]{0.25,0.50,0.50}{##1}}}
\expandafter\def\csname PY@tok@cpf\endcsname{\let\PY@it=\textit\def\PY@tc##1{\textcolor[rgb]{0.25,0.50,0.50}{##1}}}
\expandafter\def\csname PY@tok@c1\endcsname{\let\PY@it=\textit\def\PY@tc##1{\textcolor[rgb]{0.25,0.50,0.50}{##1}}}
\expandafter\def\csname PY@tok@cs\endcsname{\let\PY@it=\textit\def\PY@tc##1{\textcolor[rgb]{0.25,0.50,0.50}{##1}}}

\def\PYZbs{\char`\\}
\def\PYZus{\char`\_}
\def\PYZob{\char`\{}
\def\PYZcb{\char`\}}
\def\PYZca{\char`\^}
\def\PYZam{\char`\&}
\def\PYZlt{\char`\<}
\def\PYZgt{\char`\>}
\def\PYZsh{\char`\#}
\def\PYZpc{\char`\%}
\def\PYZdl{\char`\$}
\def\PYZhy{\char`\-}
\def\PYZsq{\char`\'}
\def\PYZdq{\char`\"}
\def\PYZti{\char`\~}
% for compatibility with earlier versions
\def\PYZat{@}
\def\PYZlb{[}
\def\PYZrb{]}
\makeatother


    % Exact colors from NB
    \definecolor{incolor}{rgb}{0.0, 0.0, 0.5}
    \definecolor{outcolor}{rgb}{0.545, 0.0, 0.0}



    
    % Prevent overflowing lines due to hard-to-break entities
    \sloppy 
    % Setup hyperref package
    \hypersetup{
      breaklinks=true,  % so long urls are correctly broken across lines
      colorlinks=true,
      urlcolor=urlcolor,
      linkcolor=linkcolor,
      citecolor=citecolor,
      }
    % Slightly bigger margins than the latex defaults
    
    \geometry{verbose,tmargin=1in,bmargin=1in,lmargin=1in,rmargin=1in}
    
    

    \begin{document}
    
    
    \maketitle
    
    

    
    CSC8635 Machine Learning Extended Technical Report

What is the need for the project?

`Training of neural networks for automated diagnosis of pigmented skin
lesions is hampered by the small size and lack of diversity of available
datasets of dermatoscopic images.' (Tschandl, Rosendahl and Kittler,
2018). In this project, I will be exploring two machine learning models
and running a comparison between them in order to choose the more
suitable model for the project domain. As there is such `a small size
and lack of diversity' of datasets surrounding this area, the project
will provide good groundwork for future projects to base off of.

`There are few comprehensive empirical studies comparing learning
algorithms' (Caruana and Niculescu-Mizil, 2006). Machine learning
comparisons are few and far between. In comparison with the small work
put towards datsets in this domain, I believe that a comparison of
machine learning technologies in this area would prove useful in
automating skin cancer diagnosis. `I am using the paper 'An empirical
comparison of supervised learning algorithms' as a basis for the
comparison, and while the paper provides a much more detailed comparison
than I am likely to need it still provides a solid background and good
practises to follow for the comparison.

    What did I do?

In order to help achieve success in this project, a methodology was
needed to follow a clear structure. KDD was decided as a suitable
methodology for this project, as a large portion of the methodology is
dedicated to data minining i.e.~machine learning in this case.

Cookie Cutter was used to help structure the project in a reproducable
format as well as providing ease of use for navigating around the
directory. A cookie cutter build for data science was found that was
excellent for the project at hand, and therefore used (Medium, 2019).

GitHub was used for version control and also allowed for backups in case
of unforeseen errors.

Due to the limited time availible for the project and often massive
runtimes of each model, only a two models were able to be compared. In
this particular instance, I compared a custom built Convolutional Neural
Network and a Virtual Geometry Group (VGG) pre-defined neural network
(Simonyan and Zisserman, 2014).

    Learning the Application Domain

In order to understand the domain and the datasets of the project, I
found it useful to read the paper from the developers of the dataset
itself, `The HAM10000dataset, a large collection of multi-source
dermatoscopic images of common pigmented skin lesions' (Tschandl,
Rosendahl and Kittler, 2018).

While an artificial neural network was already built to differentiate
between melanomas and melanocytic nevi, there was a lack of data for
both of these cell types and others.

The original datasets used to generate the datasets for this project
were discovered to have bias towards melanocytic lesions, which is
explored in my own project when plotting the dataset in the `Plotting
the Data' section.

While there do exists multi-class predictions for skin diseases, at the
time of writing the paper there were no models for dermatoscopic images,
and at the time of writing the report there are very few working models
of which the validity of the results is questionable.

    Sources used to help produce code

https://www.kaggle.com/kmader/dermatology-mnist-loading-and-processing

https://www.kaggle.com/sid321axn/step-wise-approach-cnn-model-77-0344-accuracy?fbclid=IwAR2JWtc6\_nC3PZwfg5THA1qjTx1qaLc6NllbXIAftRkZQu\_AnQ6VMgaDsIs

https://engmrk.com/vgg16-implementation-using-keras/

    \begin{Verbatim}[commandchars=\\\{\}]
{\color{incolor}In [{\color{incolor}1}]:} \PY{c+c1}{\PYZsh{} import libraries}
        \PY{k+kn}{import} \PY{n+nn}{os}
        \PY{k+kn}{import} \PY{n+nn}{numpy} \PY{k}{as} \PY{n+nn}{np}
        \PY{k+kn}{import} \PY{n+nn}{pandas} \PY{k}{as} \PY{n+nn}{pd}
        \PY{k+kn}{from} \PY{n+nn}{glob} \PY{k}{import} \PY{n}{glob}
        \PY{k+kn}{import} \PY{n+nn}{seaborn} \PY{k}{as} \PY{n+nn}{sns}
        \PY{k+kn}{import} \PY{n+nn}{matplotlib}\PY{n+nn}{.}\PY{n+nn}{pyplot} \PY{k}{as} \PY{n+nn}{plt}
        \PY{k+kn}{from} \PY{n+nn}{PIL} \PY{k}{import} \PY{n}{Image}
        \PY{k+kn}{from} \PY{n+nn}{sklearn}\PY{n+nn}{.}\PY{n+nn}{model\PYZus{}selection} \PY{k}{import} \PY{n}{train\PYZus{}test\PYZus{}split}
        \PY{k+kn}{from} \PY{n+nn}{keras}\PY{n+nn}{.}\PY{n+nn}{utils}\PY{n+nn}{.}\PY{n+nn}{np\PYZus{}utils} \PY{k}{import} \PY{n}{to\PYZus{}categorical}
        \PY{k+kn}{from} \PY{n+nn}{keras}\PY{n+nn}{.}\PY{n+nn}{models} \PY{k}{import} \PY{n}{Sequential}
        \PY{k+kn}{from} \PY{n+nn}{keras}\PY{n+nn}{.}\PY{n+nn}{layers} \PY{k}{import} \PY{n}{Dense}\PY{p}{,} \PY{n}{Dropout}\PY{p}{,} \PY{n}{Flatten}\PY{p}{,} \PY{n}{Conv2D}\PY{p}{,} \PY{n}{MaxPool2D}\PY{p}{,} \PY{n}{BatchNormalization}
        \PY{k+kn}{from} \PY{n+nn}{keras}\PY{n+nn}{.}\PY{n+nn}{optimizers} \PY{k}{import} \PY{n}{Adam}
        \PY{k+kn}{from} \PY{n+nn}{keras}\PY{n+nn}{.}\PY{n+nn}{callbacks} \PY{k}{import} \PY{n}{ReduceLROnPlateau}
        \PY{k+kn}{from} \PY{n+nn}{keras}\PY{n+nn}{.}\PY{n+nn}{preprocessing}\PY{n+nn}{.}\PY{n+nn}{image} \PY{k}{import} \PY{n}{ImageDataGenerator}
        \PY{k+kn}{from} \PY{n+nn}{sklearn}\PY{n+nn}{.}\PY{n+nn}{metrics} \PY{k}{import} \PY{n}{confusion\PYZus{}matrix}
        \PY{k+kn}{from} \PY{n+nn}{keras} \PY{k}{import} \PY{n}{applications}
        \PY{k+kn}{from} \PY{n+nn}{sklearn}\PY{n+nn}{.}\PY{n+nn}{linear\PYZus{}model} \PY{k}{import} \PY{n}{LogisticRegression}
        \PY{k+kn}{from} \PY{n+nn}{keras}\PY{n+nn}{.}\PY{n+nn}{layers} \PY{k}{import} \PY{n}{MaxPooling2D}
\end{Verbatim}


    \begin{Verbatim}[commandchars=\\\{\}]
Using TensorFlow backend.

    \end{Verbatim}

    \begin{Verbatim}[commandchars=\\\{\}]
{\color{incolor}In [{\color{incolor}2}]:} \PY{c+c1}{\PYZsh{} get working directory}
        \PY{n}{cwd} \PY{o}{=} \PY{n}{os}\PY{o}{.}\PY{n}{getcwd}\PY{p}{(}\PY{p}{)}
        \PY{n+nb}{print}\PY{p}{(}\PY{n}{cwd}\PY{p}{)}
\end{Verbatim}


    \begin{Verbatim}[commandchars=\\\{\}]
C:\textbackslash{}Users\textbackslash{}b5034806\textbackslash{}Documents\textbackslash{}Machine-Learning-Extended-Technical-Project\textbackslash{}mletp\textbackslash{}src

    \end{Verbatim}

    Due to the nature of the document, it is not possible to associate every
KDD step with the associated code, so some steps may be out of place but
all relevant information is covered at some point in the report. I have
aimed to include each step at the point where it was first used.

Creating a Target Dataset

As the dataset is quite large and the machine learning models can be
very demanding in terms of resources, it is important that only relevant
data was used. The metadata was used for the exploratory data analysis
process, but ultimately only the images were used for the machine
learning process itself. The source location

Data Cleaning and Preprocessing

While it was initially thought that there would be a large amount of
data cleaning and preprocessing, after accelerating past this step it
was noticed that data cleaning was mostly unneccesary for the project,
as explained later in the report. While it was useful to understand the
dataset further, problems like this highlight the issues of KDD.

Relevant data cleaning is discussed later in the discussion for the
models.

    \begin{Verbatim}[commandchars=\\\{\}]
{\color{incolor}In [{\color{incolor}3}]:} \PY{c+c1}{\PYZsh{} change working directory to import data}
        \PY{n}{os}\PY{o}{.}\PY{n}{chdir}\PY{p}{(}\PY{l+s+sa}{r}\PY{l+s+s2}{\PYZdq{}}\PY{l+s+s2}{C:}\PY{l+s+s2}{\PYZbs{}}\PY{l+s+s2}{Users}\PY{l+s+s2}{\PYZbs{}}\PY{l+s+s2}{b5034806}\PY{l+s+s2}{\PYZbs{}}\PY{l+s+s2}{Documents}\PY{l+s+s2}{\PYZbs{}}\PY{l+s+s2}{Machine\PYZhy{}Learning\PYZhy{}Extended\PYZhy{}Technical\PYZhy{}Project}\PY{l+s+s2}{\PYZbs{}}\PY{l+s+s2}{mletp}\PY{l+s+s2}{\PYZbs{}}\PY{l+s+s2}{data}\PY{l+s+s2}{\PYZdq{}}\PY{p}{)}
\end{Verbatim}


    \begin{Verbatim}[commandchars=\\\{\}]
{\color{incolor}In [{\color{incolor}4}]:} \PY{n}{metadata\PYZus{}df} \PY{o}{=} \PY{n}{pd}\PY{o}{.}\PY{n}{read\PYZus{}csv}\PY{p}{(}\PY{l+s+s2}{\PYZdq{}}\PY{l+s+s2}{HAM10000\PYZus{}metadata.csv}\PY{l+s+s2}{\PYZdq{}}\PY{p}{)}
\end{Verbatim}


    Image paths were assigned to the metadata dataframe in order to provide
a `merge' and connect the data.

    \begin{Verbatim}[commandchars=\\\{\}]
{\color{incolor}In [{\color{incolor}5}]:} \PY{c+c1}{\PYZsh{} dictionary for image paths}
        \PY{n}{imageid\PYZus{}path\PYZus{}dict} \PY{o}{=} \PY{p}{\PYZob{}}\PY{n}{os}\PY{o}{.}\PY{n}{path}\PY{o}{.}\PY{n}{splitext}\PY{p}{(}\PY{n}{os}\PY{o}{.}\PY{n}{path}\PY{o}{.}\PY{n}{basename}\PY{p}{(}\PY{n}{x}\PY{p}{)}\PY{p}{)}\PY{p}{[}\PY{l+m+mi}{0}\PY{p}{]}\PY{p}{:} \PY{n}{x}
                             \PY{k}{for} \PY{n}{x} \PY{o+ow}{in} \PY{n}{glob}\PY{p}{(}\PY{n}{os}\PY{o}{.}\PY{n}{path}\PY{o}{.}\PY{n}{join}\PY{p}{(}\PY{l+s+s1}{\PYZsq{}}\PY{l+s+s1}{*}\PY{l+s+s1}{\PYZsq{}}\PY{p}{,} \PY{l+s+s1}{\PYZsq{}}\PY{l+s+s1}{*.jpg}\PY{l+s+s1}{\PYZsq{}}\PY{p}{)}\PY{p}{)}\PY{p}{\PYZcb{}}
\end{Verbatim}


    \begin{Verbatim}[commandchars=\\\{\}]
{\color{incolor}In [{\color{incolor}6}]:} \PY{c+c1}{\PYZsh{} change working directory back}
        \PY{n}{os}\PY{o}{.}\PY{n}{chdir}\PY{p}{(}\PY{l+s+sa}{r}\PY{l+s+s2}{\PYZdq{}}\PY{l+s+s2}{C:}\PY{l+s+s2}{\PYZbs{}}\PY{l+s+s2}{Users}\PY{l+s+s2}{\PYZbs{}}\PY{l+s+s2}{b5034806}\PY{l+s+s2}{\PYZbs{}}\PY{l+s+s2}{Documents}\PY{l+s+s2}{\PYZbs{}}\PY{l+s+s2}{Machine\PYZhy{}Learning\PYZhy{}Extended\PYZhy{}Technical\PYZhy{}Project}\PY{l+s+s2}{\PYZbs{}}\PY{l+s+s2}{mletp}\PY{l+s+s2}{\PYZbs{}}\PY{l+s+s2}{src}\PY{l+s+s2}{\PYZdq{}}\PY{p}{)}
\end{Verbatim}


    A dictionary was produced for readability during the exploratory data
analysis phase and general ease of use throughout the process.

    \begin{Verbatim}[commandchars=\\\{\}]
{\color{incolor}In [{\color{incolor}7}]:} \PY{c+c1}{\PYZsh{} dictionary to rename cell types for readability}
        \PY{n}{lesion\PYZus{}type\PYZus{}dict} \PY{o}{=} \PY{p}{\PYZob{}}
            \PY{l+s+s1}{\PYZsq{}}\PY{l+s+s1}{nv}\PY{l+s+s1}{\PYZsq{}}\PY{p}{:} \PY{l+s+s1}{\PYZsq{}}\PY{l+s+s1}{Melanocytic Nevi}\PY{l+s+s1}{\PYZsq{}}\PY{p}{,}
            \PY{l+s+s1}{\PYZsq{}}\PY{l+s+s1}{mel}\PY{l+s+s1}{\PYZsq{}}\PY{p}{:} \PY{l+s+s1}{\PYZsq{}}\PY{l+s+s1}{Melanoma}\PY{l+s+s1}{\PYZsq{}}\PY{p}{,}
            \PY{l+s+s1}{\PYZsq{}}\PY{l+s+s1}{bkl}\PY{l+s+s1}{\PYZsq{}}\PY{p}{:} \PY{l+s+s1}{\PYZsq{}}\PY{l+s+s1}{Benign Keratosis\PYZhy{}like Lesions }\PY{l+s+s1}{\PYZsq{}}\PY{p}{,}
            \PY{l+s+s1}{\PYZsq{}}\PY{l+s+s1}{bcc}\PY{l+s+s1}{\PYZsq{}}\PY{p}{:} \PY{l+s+s1}{\PYZsq{}}\PY{l+s+s1}{Basal Cell Carcinoma}\PY{l+s+s1}{\PYZsq{}}\PY{p}{,}
            \PY{l+s+s1}{\PYZsq{}}\PY{l+s+s1}{akiec}\PY{l+s+s1}{\PYZsq{}}\PY{p}{:} \PY{l+s+s1}{\PYZsq{}}\PY{l+s+s1}{Bowens Disease}\PY{l+s+s1}{\PYZsq{}}\PY{p}{,}
            \PY{l+s+s1}{\PYZsq{}}\PY{l+s+s1}{vasc}\PY{l+s+s1}{\PYZsq{}}\PY{p}{:} \PY{l+s+s1}{\PYZsq{}}\PY{l+s+s1}{Vascular Lesions}\PY{l+s+s1}{\PYZsq{}}\PY{p}{,}
            \PY{l+s+s1}{\PYZsq{}}\PY{l+s+s1}{df}\PY{l+s+s1}{\PYZsq{}}\PY{p}{:} \PY{l+s+s1}{\PYZsq{}}\PY{l+s+s1}{Dermatofibroma}\PY{l+s+s1}{\PYZsq{}}
        \PY{p}{\PYZcb{}}
\end{Verbatim}


    Cells were assigned an ID to be used as a response variable later in the
project.

    \begin{Verbatim}[commandchars=\\\{\}]
{\color{incolor}In [{\color{incolor}8}]:} \PY{c+c1}{\PYZsh{} add column for path of image}
        \PY{n}{metadata\PYZus{}df}\PY{p}{[}\PY{l+s+s1}{\PYZsq{}}\PY{l+s+s1}{path}\PY{l+s+s1}{\PYZsq{}}\PY{p}{]} \PY{o}{=} \PY{n}{metadata\PYZus{}df}\PY{p}{[}\PY{l+s+s1}{\PYZsq{}}\PY{l+s+s1}{image\PYZus{}id}\PY{l+s+s1}{\PYZsq{}}\PY{p}{]}\PY{o}{.}\PY{n}{map}\PY{p}{(}\PY{n}{imageid\PYZus{}path\PYZus{}dict}\PY{o}{.}\PY{n}{get}\PY{p}{)}
        \PY{c+c1}{\PYZsh{} add column for cell types, for readability}
        \PY{n}{metadata\PYZus{}df}\PY{p}{[}\PY{l+s+s1}{\PYZsq{}}\PY{l+s+s1}{cell\PYZus{}type}\PY{l+s+s1}{\PYZsq{}}\PY{p}{]} \PY{o}{=} \PY{n}{metadata\PYZus{}df}\PY{p}{[}\PY{l+s+s1}{\PYZsq{}}\PY{l+s+s1}{dx}\PY{l+s+s1}{\PYZsq{}}\PY{p}{]}\PY{o}{.}\PY{n}{map}\PY{p}{(}\PY{n}{lesion\PYZus{}type\PYZus{}dict}\PY{o}{.}\PY{n}{get}\PY{p}{)}
        \PY{c+c1}{\PYZsh{} assign each cell type an ID and add it to a column}
        \PY{n}{metadata\PYZus{}df}\PY{p}{[}\PY{l+s+s1}{\PYZsq{}}\PY{l+s+s1}{cell\PYZus{}type\PYZus{}id}\PY{l+s+s1}{\PYZsq{}}\PY{p}{]} \PY{o}{=} \PY{n}{pd}\PY{o}{.}\PY{n}{Categorical}\PY{p}{(}\PY{n}{metadata\PYZus{}df}\PY{p}{[}\PY{l+s+s1}{\PYZsq{}}\PY{l+s+s1}{cell\PYZus{}type}\PY{l+s+s1}{\PYZsq{}}\PY{p}{]}\PY{p}{)}\PY{o}{.}\PY{n}{codes}
\end{Verbatim}


    Some introductory exploratory data analysis was performed in order to
get a feel and general understanding for the data. This included use of
the head, tail, and columns functions.

    \begin{Verbatim}[commandchars=\\\{\}]
{\color{incolor}In [{\color{incolor}9}]:} \PY{c+c1}{\PYZsh{} print first 5 values of dataframe}
        \PY{n+nb}{print}\PY{p}{(}\PY{n}{metadata\PYZus{}df}\PY{o}{.}\PY{n}{head}\PY{p}{(}\PY{l+m+mi}{5}\PY{p}{)}\PY{p}{)}
\end{Verbatim}


    \begin{Verbatim}[commandchars=\\\{\}]
     lesion\_id      image\_id   dx dx\_type   age   sex localization  \textbackslash{}
0  HAM\_0000118  ISIC\_0027419  bkl   histo  80.0  male        scalp   
1  HAM\_0000118  ISIC\_0025030  bkl   histo  80.0  male        scalp   
2  HAM\_0002730  ISIC\_0026769  bkl   histo  80.0  male        scalp   
3  HAM\_0002730  ISIC\_0025661  bkl   histo  80.0  male        scalp   
4  HAM\_0001466  ISIC\_0031633  bkl   histo  75.0  male          ear   

                                      path                       cell\_type  \textbackslash{}
0  HAM10000\_images\_part\_1\textbackslash{}ISIC\_0027419.jpg  Benign Keratosis-like Lesions    
1  HAM10000\_images\_part\_1\textbackslash{}ISIC\_0025030.jpg  Benign Keratosis-like Lesions    
2  HAM10000\_images\_part\_1\textbackslash{}ISIC\_0026769.jpg  Benign Keratosis-like Lesions    
3  HAM10000\_images\_part\_1\textbackslash{}ISIC\_0025661.jpg  Benign Keratosis-like Lesions    
4  HAM10000\_images\_part\_2\textbackslash{}ISIC\_0031633.jpg  Benign Keratosis-like Lesions    

   cell\_type\_id  
0             1  
1             1  
2             1  
3             1  
4             1  

    \end{Verbatim}

    \begin{Verbatim}[commandchars=\\\{\}]
{\color{incolor}In [{\color{incolor}10}]:} \PY{c+c1}{\PYZsh{} print first 5 values of dataframe}
         \PY{n+nb}{print}\PY{p}{(}\PY{n}{metadata\PYZus{}df}\PY{o}{.}\PY{n}{tail}\PY{p}{(}\PY{l+m+mi}{5}\PY{p}{)}\PY{p}{)}
\end{Verbatim}


    \begin{Verbatim}[commandchars=\\\{\}]
         lesion\_id      image\_id     dx dx\_type   age     sex localization  \textbackslash{}
10010  HAM\_0002867  ISIC\_0033084  akiec   histo  40.0    male      abdomen   
10011  HAM\_0002867  ISIC\_0033550  akiec   histo  40.0    male      abdomen   
10012  HAM\_0002867  ISIC\_0033536  akiec   histo  40.0    male      abdomen   
10013  HAM\_0000239  ISIC\_0032854  akiec   histo  80.0    male         face   
10014  HAM\_0003521  ISIC\_0032258    mel   histo  70.0  female         back   

                                          path       cell\_type  cell\_type\_id  
10010  HAM10000\_images\_part\_2\textbackslash{}ISIC\_0033084.jpg  Bowens Disease             2  
10011  HAM10000\_images\_part\_2\textbackslash{}ISIC\_0033550.jpg  Bowens Disease             2  
10012  HAM10000\_images\_part\_2\textbackslash{}ISIC\_0033536.jpg  Bowens Disease             2  
10013  HAM10000\_images\_part\_2\textbackslash{}ISIC\_0032854.jpg  Bowens Disease             2  
10014  HAM10000\_images\_part\_2\textbackslash{}ISIC\_0032258.jpg        Melanoma             5  

    \end{Verbatim}

    This allows us to see the first and last 5 values of the metadata
dataset, and get a general understanding of how the data is formatted.

    \begin{Verbatim}[commandchars=\\\{\}]
{\color{incolor}In [{\color{incolor}11}]:} \PY{c+c1}{\PYZsh{} print column names}
         \PY{n+nb}{print}\PY{p}{(}\PY{n}{metadata\PYZus{}df}\PY{o}{.}\PY{n}{columns}\PY{p}{)}
\end{Verbatim}


    \begin{Verbatim}[commandchars=\\\{\}]
Index(['lesion\_id', 'image\_id', 'dx', 'dx\_type', 'age', 'sex', 'localization',
       'path', 'cell\_type', 'cell\_type\_id'],
      dtype='object')

    \end{Verbatim}

    \begin{Verbatim}[commandchars=\\\{\}]
{\color{incolor}In [{\color{incolor} }]:} \PY{c+c1}{\PYZsh{} print columns and their data type}
        \PY{n+nb}{print}\PY{p}{(}\PY{n}{metadata\PYZus{}df}\PY{o}{.}\PY{n}{dtypes}\PY{p}{)}
\end{Verbatim}


    \begin{Verbatim}[commandchars=\\\{\}]
{\color{incolor}In [{\color{incolor}12}]:} \PY{c+c1}{\PYZsh{} summary of numeric values}
         \PY{n+nb}{print}\PY{p}{(}\PY{n}{metadata\PYZus{}df}\PY{o}{.}\PY{n}{describe}\PY{p}{(}\PY{p}{)}\PY{p}{)}
\end{Verbatim}


    \begin{Verbatim}[commandchars=\\\{\}]
               age  cell\_type\_id
count  9958.000000  10015.000000
mean     51.863828      3.528208
std      16.968614      1.377071
min       0.000000      0.000000
25\%      40.000000      4.000000
50\%      50.000000      4.000000
75\%      65.000000      4.000000
max      85.000000      6.000000

    \end{Verbatim}

    While the mean age is calculated as 52, as the ages in the dataset are
given in intervals of 5, this may not actually be a true representation
of the ages. However, for our purposes with the dataset as long as the
age is kept consistent throughout this is not a large issue, although in
order to get a more precise model in a future project it may help to
give the exact ages.

    Despite images only being used for the final model, data cleaning was
still used due to the methodologys structure and general good practise.
While it was good practise, for my particular models these steps were
irrelevant for training the models and optimising accuracy. The
documentation for this data cleaning and commented code has been left in
the report for reproducability. This also provides a basis for future
work on the project if someone chose to include the meta data in their
model, which may be beneficial for comparing image classification and
extracting features from the image and classifying using the data.

For data cleaning, it is often important to deal with null/NA values
from data to remove redundant rows from the dataset. In this situation,
as age was the only column with null values I opted to replace them with
the mean value of the age column. In order to improve accuracy further,
it may be worthwhile in a future project to look at other factors that
my affect age and calculate the age based of of that e.g.~are certain
ages more likely to have a certain cell type? While not null values, I
removed any rows that had unknown values (prevalent in the sex and
localization columns) to maximise information gain.

    \begin{Verbatim}[commandchars=\\\{\}]
{\color{incolor}In [{\color{incolor}13}]:} \PY{c+c1}{\PYZsh{} gives us data types and how many values of each column are non\PYZhy{}null}
         \PY{n+nb}{print}\PY{p}{(}\PY{n}{metadata\PYZus{}df}\PY{o}{.}\PY{n}{info}\PY{p}{(}\PY{p}{)}\PY{p}{)}
\end{Verbatim}


    \begin{Verbatim}[commandchars=\\\{\}]
<class 'pandas.core.frame.DataFrame'>
RangeIndex: 10015 entries, 0 to 10014
Data columns (total 10 columns):
lesion\_id       10015 non-null object
image\_id        10015 non-null object
dx              10015 non-null object
dx\_type         10015 non-null object
age             9958 non-null float64
sex             10015 non-null object
localization    10015 non-null object
path            10015 non-null object
cell\_type       10015 non-null object
cell\_type\_id    10015 non-null int8
dtypes: float64(1), int8(1), object(8)
memory usage: 714.0+ KB
None

    \end{Verbatim}

    \begin{Verbatim}[commandchars=\\\{\}]
{\color{incolor}In [{\color{incolor}14}]:} \PY{c+c1}{\PYZsh{} fill null values of age with the mean value of the age column}
         \PY{c+c1}{\PYZsh{}metadata\PYZus{}df[\PYZdq{}age\PYZdq{}] = metadata\PYZus{}df[\PYZdq{}age\PYZdq{}].fillna(int(metadata\PYZus{}df[\PYZdq{}age\PYZdq{}].mean()))}
\end{Verbatim}


    \begin{Verbatim}[commandchars=\\\{\}]
{\color{incolor}In [{\color{incolor}15}]:} \PY{c+c1}{\PYZsh{} display unique values in cell type column}
         \PY{n+nb}{print}\PY{p}{(}\PY{n}{metadata\PYZus{}df}\PY{o}{.}\PY{n}{dx}\PY{o}{.}\PY{n}{unique}\PY{p}{(}\PY{p}{)}\PY{p}{)}
\end{Verbatim}


    \begin{Verbatim}[commandchars=\\\{\}]
['bkl' 'nv' 'df' 'mel' 'vasc' 'bcc' 'akiec']

    \end{Verbatim}

    \begin{Verbatim}[commandchars=\\\{\}]
{\color{incolor}In [{\color{incolor}16}]:} \PY{c+c1}{\PYZsh{} display unique values in diagnosis column}
         \PY{n+nb}{print}\PY{p}{(}\PY{n}{metadata\PYZus{}df}\PY{o}{.}\PY{n}{dx\PYZus{}type}\PY{o}{.}\PY{n}{unique}\PY{p}{(}\PY{p}{)}\PY{p}{)}
\end{Verbatim}


    \begin{Verbatim}[commandchars=\\\{\}]
['histo' 'consensus' 'confocal' 'follow\_up']

    \end{Verbatim}

    \begin{Verbatim}[commandchars=\\\{\}]
{\color{incolor}In [{\color{incolor}17}]:} \PY{c+c1}{\PYZsh{} display unique values in sex column}
         \PY{n+nb}{print}\PY{p}{(}\PY{n}{metadata\PYZus{}df}\PY{o}{.}\PY{n}{sex}\PY{o}{.}\PY{n}{unique}\PY{p}{(}\PY{p}{)}\PY{p}{)}
\end{Verbatim}


    \begin{Verbatim}[commandchars=\\\{\}]
['male' 'female' 'unknown']

    \end{Verbatim}

    \begin{Verbatim}[commandchars=\\\{\}]
{\color{incolor}In [{\color{incolor}18}]:} \PY{c+c1}{\PYZsh{} drop rows where sex = unknown}
         \PY{n}{metadata\PYZus{}df} \PY{o}{=} \PY{n}{metadata\PYZus{}df}\PY{p}{[}\PY{n}{metadata\PYZus{}df}\PY{o}{.}\PY{n}{sex}\PY{o}{!=}\PY{l+s+s1}{\PYZsq{}}\PY{l+s+s1}{unknown}\PY{l+s+s1}{\PYZsq{}}\PY{p}{]}
\end{Verbatim}


    \begin{Verbatim}[commandchars=\\\{\}]
{\color{incolor}In [{\color{incolor}19}]:} \PY{c+c1}{\PYZsh{} display unique values in sex column}
         \PY{n+nb}{print}\PY{p}{(}\PY{n}{metadata\PYZus{}df}\PY{o}{.}\PY{n}{localization}\PY{o}{.}\PY{n}{unique}\PY{p}{(}\PY{p}{)}\PY{p}{)}
\end{Verbatim}


    \begin{Verbatim}[commandchars=\\\{\}]
['scalp' 'ear' 'face' 'back' 'trunk' 'chest' 'upper extremity' 'abdomen'
 'unknown' 'lower extremity' 'genital' 'neck' 'hand' 'foot' 'acral']

    \end{Verbatim}

    \begin{Verbatim}[commandchars=\\\{\}]
{\color{incolor}In [{\color{incolor}20}]:} \PY{c+c1}{\PYZsh{} drop rows where localization = unknown}
         \PY{n}{metadata\PYZus{}df} \PY{o}{=} \PY{n}{metadata\PYZus{}df}\PY{p}{[}\PY{n}{metadata\PYZus{}df}\PY{o}{.}\PY{n}{localization}\PY{o}{!=}\PY{l+s+s1}{\PYZsq{}}\PY{l+s+s1}{unknown}\PY{l+s+s1}{\PYZsq{}}\PY{p}{]}
\end{Verbatim}


    Plotting the Data

Data was plotted to allow further exploration in the dataset as well as
giving the oppurtunity to spot any patterns that the dataset may
contain.

    \begin{Verbatim}[commandchars=\\\{\}]
{\color{incolor}In [{\color{incolor}21}]:} \PY{c+c1}{\PYZsh{} countplot of cell type}
         \PY{n}{plt}\PY{o}{.}\PY{n}{figure}\PY{p}{(}\PY{n}{figsize}\PY{o}{=}\PY{p}{(}\PY{l+m+mi}{20}\PY{p}{,} \PY{l+m+mi}{10}\PY{p}{)}\PY{p}{)}
         
         \PY{n}{sns}\PY{o}{.}\PY{n}{countplot}\PY{p}{(}\PY{n}{x} \PY{o}{=} \PY{l+s+s1}{\PYZsq{}}\PY{l+s+s1}{cell\PYZus{}type}\PY{l+s+s1}{\PYZsq{}}\PY{p}{,}
                       \PY{n}{data} \PY{o}{=} \PY{n}{metadata\PYZus{}df}\PY{p}{,}
                       \PY{n}{order} \PY{o}{=} \PY{n}{metadata\PYZus{}df}\PY{p}{[}\PY{l+s+s1}{\PYZsq{}}\PY{l+s+s1}{cell\PYZus{}type}\PY{l+s+s1}{\PYZsq{}}\PY{p}{]}\PY{o}{.}\PY{n}{value\PYZus{}counts}\PY{p}{(}\PY{p}{)}\PY{o}{.}\PY{n}{index}\PY{p}{)}
         \PY{n}{plt}\PY{o}{.}\PY{n}{title}\PY{p}{(}\PY{l+s+s2}{\PYZdq{}}\PY{l+s+s2}{Countplot}\PY{l+s+s2}{\PYZdq{}}\PY{p}{)}
         \PY{n}{plt}\PY{o}{.}\PY{n}{show}\PY{p}{(}\PY{p}{)}
\end{Verbatim}


    \begin{center}
    \adjustimage{max size={0.9\linewidth}{0.9\paperheight}}{output_34_0.png}
    \end{center}
    { \hspace*{\fill} \\}
    
    \begin{Verbatim}[commandchars=\\\{\}]
{\color{incolor}In [{\color{incolor}22}]:} \PY{n}{metadata\PYZus{}df}\PY{o}{.}\PY{n}{groupby}\PY{p}{(}\PY{l+s+s1}{\PYZsq{}}\PY{l+s+s1}{cell\PYZus{}type}\PY{l+s+s1}{\PYZsq{}}\PY{p}{)}\PY{o}{.}\PY{n}{count}\PY{p}{(}\PY{p}{)}
\end{Verbatim}


\begin{Verbatim}[commandchars=\\\{\}]
{\color{outcolor}Out[{\color{outcolor}22}]:}                                 lesion\_id  image\_id    dx  dx\_type   age  \textbackslash{}
         cell\_type                                                                  
         Basal Cell Carcinoma                  514       514   514      514   514   
         Benign Keratosis-like Lesions        1099      1099  1099     1099  1089   
         Bowens Disease                        327       327   327      327   327   
         Dermatofibroma                        115       115   115      115   115   
         Melanocytic Nevi                     6705      6705  6705     6705  6660   
         Melanoma                             1113      1113  1113     1113  1111   
         Vascular Lesions                      142       142   142      142   142   
         
                                          sex  localization  path  cell\_type\_id  
         cell\_type                                                               
         Basal Cell Carcinoma             514           514   514           514  
         Benign Keratosis-like Lesions   1099          1099  1099          1099  
         Bowens Disease                   327           327   327           327  
         Dermatofibroma                   115           115   115           115  
         Melanocytic Nevi                6705          6705  6705          6705  
         Melanoma                        1113          1113  1113          1113  
         Vascular Lesions                 142           142   142           142  
\end{Verbatim}
            
    Here we can see that there are a much larger amount of Melanocytic Nevi
cells than any other type, with approximately 6 times more values then
the next most frequent cell, melanoma. The least frequent cell type is
Dermatofibroma. This means that there is a huge range of 6384. This is
important to consider for the machine learning models, as it may lead to
overfitting. For example, if the model was incorrect and predicted every
image as Melanocytic Nevi, it would result in a high accuracy for the
training data but low accuracy for the test data. This also stresses the
importance of using a confusion matrix, as it allows me to verify the
validity of my results.

    \begin{Verbatim}[commandchars=\\\{\}]
{\color{incolor}In [{\color{incolor}23}]:} \PY{c+c1}{\PYZsh{} countplot of diagnosis}
         \PY{n}{plt}\PY{o}{.}\PY{n}{figure}\PY{p}{(}\PY{n}{figsize}\PY{o}{=}\PY{p}{(}\PY{l+m+mi}{20}\PY{p}{,} \PY{l+m+mi}{10}\PY{p}{)}\PY{p}{)}
         
         \PY{n}{sns}\PY{o}{.}\PY{n}{countplot}\PY{p}{(}\PY{n}{x} \PY{o}{=} \PY{l+s+s1}{\PYZsq{}}\PY{l+s+s1}{dx\PYZus{}type}\PY{l+s+s1}{\PYZsq{}}\PY{p}{,}
                       \PY{n}{data} \PY{o}{=} \PY{n}{metadata\PYZus{}df}\PY{p}{,}
                       \PY{n}{order} \PY{o}{=} \PY{n}{metadata\PYZus{}df}\PY{p}{[}\PY{l+s+s1}{\PYZsq{}}\PY{l+s+s1}{dx\PYZus{}type}\PY{l+s+s1}{\PYZsq{}}\PY{p}{]}\PY{o}{.}\PY{n}{value\PYZus{}counts}\PY{p}{(}\PY{p}{)}\PY{o}{.}\PY{n}{index}\PY{p}{)}
         \PY{n}{plt}\PY{o}{.}\PY{n}{title}\PY{p}{(}\PY{l+s+s2}{\PYZdq{}}\PY{l+s+s2}{Countplot}\PY{l+s+s2}{\PYZdq{}}\PY{p}{)}
         \PY{n}{plt}\PY{o}{.}\PY{n}{show}\PY{p}{(}\PY{p}{)}
\end{Verbatim}


    \begin{center}
    \adjustimage{max size={0.9\linewidth}{0.9\paperheight}}{output_37_0.png}
    \end{center}
    { \hspace*{\fill} \\}
    
    A countplot of the diagnosis type was produced, which allows us to see
how the cells in the images were diagnosed. Although this will not
necessarily affect our results, it is worth noting that this gives us
some insights into how the cancer is diagnosed and potential false
diagnoses.

    \begin{Verbatim}[commandchars=\\\{\}]
{\color{incolor}In [{\color{incolor}24}]:} \PY{c+c1}{\PYZsh{} histogram of age}
         \PY{n}{plt}\PY{o}{.}\PY{n}{figure}\PY{p}{(}\PY{n}{figsize}\PY{o}{=}\PY{p}{(}\PY{l+m+mi}{20}\PY{p}{,} \PY{l+m+mi}{10}\PY{p}{)}\PY{p}{)}
         
         \PY{n}{plt}\PY{o}{.}\PY{n}{hist}\PY{p}{(}\PY{n}{metadata\PYZus{}df}\PY{p}{[}\PY{l+s+s1}{\PYZsq{}}\PY{l+s+s1}{age}\PY{l+s+s1}{\PYZsq{}}\PY{p}{]}\PY{p}{,} \PY{n}{bins} \PY{o}{=} \PY{l+m+mi}{20}\PY{p}{,} \PY{n}{histtype}\PY{o}{=}\PY{l+s+s1}{\PYZsq{}}\PY{l+s+s1}{bar}\PY{l+s+s1}{\PYZsq{}}\PY{p}{,} \PY{n}{ec}\PY{o}{=}\PY{l+s+s1}{\PYZsq{}}\PY{l+s+s1}{black}\PY{l+s+s1}{\PYZsq{}}\PY{p}{)}
         \PY{n}{plt}\PY{o}{.}\PY{n}{title}\PY{p}{(}\PY{l+s+s2}{\PYZdq{}}\PY{l+s+s2}{Histogram}\PY{l+s+s2}{\PYZdq{}}\PY{p}{)}
         \PY{n}{plt}\PY{o}{.}\PY{n}{show}\PY{p}{(}\PY{p}{)}
\end{Verbatim}


    \begin{Verbatim}[commandchars=\\\{\}]
C:\textbackslash{}Users\textbackslash{}b5034806\textbackslash{}AppData\textbackslash{}Local\textbackslash{}Continuum\textbackslash{}anaconda3\textbackslash{}envs\textbackslash{}MLEnv3\textbackslash{}lib\textbackslash{}site-packages\textbackslash{}numpy\textbackslash{}lib\textbackslash{}histograms.py:754: RuntimeWarning: invalid value encountered in greater\_equal
  keep = (tmp\_a >= first\_edge)
C:\textbackslash{}Users\textbackslash{}b5034806\textbackslash{}AppData\textbackslash{}Local\textbackslash{}Continuum\textbackslash{}anaconda3\textbackslash{}envs\textbackslash{}MLEnv3\textbackslash{}lib\textbackslash{}site-packages\textbackslash{}numpy\textbackslash{}lib\textbackslash{}histograms.py:755: RuntimeWarning: invalid value encountered in less\_equal
  keep \&= (tmp\_a <= last\_edge)

    \end{Verbatim}

    \begin{center}
    \adjustimage{max size={0.9\linewidth}{0.9\paperheight}}{output_39_1.png}
    \end{center}
    { \hspace*{\fill} \\}
    
    To see the age distribution of the dataset, a histogram was produced. As
we can see, the majority of people are in the 40-60 range, with very few
people under the age of 20 included in the dataset.

    \begin{Verbatim}[commandchars=\\\{\}]
{\color{incolor}In [{\color{incolor}25}]:} \PY{c+c1}{\PYZsh{} countplot of sex}
         \PY{n}{plt}\PY{o}{.}\PY{n}{figure}\PY{p}{(}\PY{n}{figsize}\PY{o}{=}\PY{p}{(}\PY{l+m+mi}{20}\PY{p}{,} \PY{l+m+mi}{10}\PY{p}{)}\PY{p}{)}
         
         \PY{n}{sns}\PY{o}{.}\PY{n}{countplot}\PY{p}{(}\PY{n}{x} \PY{o}{=} \PY{l+s+s1}{\PYZsq{}}\PY{l+s+s1}{sex}\PY{l+s+s1}{\PYZsq{}}\PY{p}{,}
                       \PY{n}{data} \PY{o}{=} \PY{n}{metadata\PYZus{}df}\PY{p}{,}
                       \PY{n}{order} \PY{o}{=} \PY{n}{metadata\PYZus{}df}\PY{p}{[}\PY{l+s+s1}{\PYZsq{}}\PY{l+s+s1}{sex}\PY{l+s+s1}{\PYZsq{}}\PY{p}{]}\PY{o}{.}\PY{n}{value\PYZus{}counts}\PY{p}{(}\PY{p}{)}\PY{o}{.}\PY{n}{index}\PY{p}{)}
         \PY{n}{plt}\PY{o}{.}\PY{n}{title}\PY{p}{(}\PY{l+s+s2}{\PYZdq{}}\PY{l+s+s2}{Countplot}\PY{l+s+s2}{\PYZdq{}}\PY{p}{)}
         \PY{n}{plt}\PY{o}{.}\PY{n}{show}\PY{p}{(}\PY{p}{)}
\end{Verbatim}


    \begin{center}
    \adjustimage{max size={0.9\linewidth}{0.9\paperheight}}{output_41_0.png}
    \end{center}
    { \hspace*{\fill} \\}
    
    \begin{Verbatim}[commandchars=\\\{\}]
{\color{incolor}In [{\color{incolor}26}]:} \PY{c+c1}{\PYZsh{} countplot of loalization}
         \PY{n}{plt}\PY{o}{.}\PY{n}{figure}\PY{p}{(}\PY{n}{figsize}\PY{o}{=}\PY{p}{(}\PY{l+m+mi}{20}\PY{p}{,} \PY{l+m+mi}{10}\PY{p}{)}\PY{p}{)}
         
         \PY{n}{sns}\PY{o}{.}\PY{n}{countplot}\PY{p}{(}\PY{n}{x} \PY{o}{=} \PY{l+s+s1}{\PYZsq{}}\PY{l+s+s1}{localization}\PY{l+s+s1}{\PYZsq{}}\PY{p}{,}
                       \PY{n}{data} \PY{o}{=} \PY{n}{metadata\PYZus{}df}\PY{p}{,}
                       \PY{n}{order} \PY{o}{=} \PY{n}{metadata\PYZus{}df}\PY{p}{[}\PY{l+s+s1}{\PYZsq{}}\PY{l+s+s1}{localization}\PY{l+s+s1}{\PYZsq{}}\PY{p}{]}\PY{o}{.}\PY{n}{value\PYZus{}counts}\PY{p}{(}\PY{p}{)}\PY{o}{.}\PY{n}{index}\PY{p}{)}
         \PY{n}{plt}\PY{o}{.}\PY{n}{title}\PY{p}{(}\PY{l+s+s2}{\PYZdq{}}\PY{l+s+s2}{Countplot}\PY{l+s+s2}{\PYZdq{}}\PY{p}{)}
         \PY{n}{plt}\PY{o}{.}\PY{n}{show}\PY{p}{(}\PY{p}{)}
\end{Verbatim}


    \begin{center}
    \adjustimage{max size={0.9\linewidth}{0.9\paperheight}}{output_42_0.png}
    \end{center}
    { \hspace*{\fill} \\}
    
    Data Reduction and Projection

As neural networks are a natural choice for images and models excel
without extracting the features, the metadata was not included for the
final model and only the images were used.

Choosing Function of Data Mining

Using the background reading, it was clear that this is a classification
problem and that the models should predict the cell type, similarly to
the old artificial neural network discussed in the `Learning the
application domain' section. The models will essentially be an expansion
on that, with five more cell types.

Choosing the data mining algorithm

I have decided that CNN and VGG will be a natural fit for this project,
as they are both convolutional neural networks which are a natural
solution for image classification, and it allows me to compare a custom
built model with a pre-defined model.

Data mining

The data mining process is discussed in depth throughout the discussion
of the models:

    Convolutional Neural Network

CNNs are based upon neural networks in that they are based upon neurons.
However, they differ from neural networks by taking in a multi chaneled
image, which is suitable for our dataset as the images are RGB values
which is read in as a 600x600x3 array, which equates to 3 chanels. CNNs
are composed of convolutional, nonlinear, pooling and fully connected
layers. The convolution layer works by convolving a filter around the
image and computing element wise multiplications based upon the
receptive field, which is summed up and repeated for each location on
the image, which results in the feature map. This is how the model
identified the features. The fully connected layer works by taking the
high level features and associates these features with a class. In order
to train a CNN, back propogation is used.

    While all of the images are the same size and image resizing is not
necessary, due to the timescale of the project it was essential for me
to make the image sizes smaller to allow for faster run times, although
this ultimately could affect the accuracy of the models due to a loss of
data in the images. I have therefore tried to balance run time and
accuracy, and resized the images to 1/3 of the original size. In the
future, given a larger timescale/ better resources I would like to
rexplore the dataset using the original image sizes and compare the
accuracies and see if there is any significant differences between the
original and rescaled images. While not directly applicable to our
dataset, it is also important to note that padding is inefficient for
classification problems, as it may cost some epochs for the neural
network to calculate that there is no correlation in the black pixels
using gradient descent (Howard, 2019).

    \begin{Verbatim}[commandchars=\\\{\}]
{\color{incolor}In [{\color{incolor}87}]:} \PY{c+c1}{\PYZsh{} change working directory for resizing}
         \PY{n}{os}\PY{o}{.}\PY{n}{chdir}\PY{p}{(}\PY{l+s+sa}{r}\PY{l+s+s2}{\PYZdq{}}\PY{l+s+s2}{C:}\PY{l+s+s2}{\PYZbs{}}\PY{l+s+s2}{Users}\PY{l+s+s2}{\PYZbs{}}\PY{l+s+s2}{b5034806}\PY{l+s+s2}{\PYZbs{}}\PY{l+s+s2}{Documents}\PY{l+s+s2}{\PYZbs{}}\PY{l+s+s2}{Machine\PYZhy{}Learning\PYZhy{}Extended\PYZhy{}Technical\PYZhy{}Project}\PY{l+s+s2}{\PYZbs{}}\PY{l+s+s2}{mletp}\PY{l+s+s2}{\PYZbs{}}\PY{l+s+s2}{data}\PY{l+s+s2}{\PYZdq{}}\PY{p}{)}
         
         \PY{n}{metadata\PYZus{}df}\PY{p}{[}\PY{l+s+s1}{\PYZsq{}}\PY{l+s+s1}{image}\PY{l+s+s1}{\PYZsq{}}\PY{p}{]} \PY{o}{=} \PY{n}{metadata\PYZus{}df}\PY{p}{[}\PY{l+s+s1}{\PYZsq{}}\PY{l+s+s1}{path}\PY{l+s+s1}{\PYZsq{}}\PY{p}{]}\PY{o}{.}\PY{n}{map}\PY{p}{(}\PY{k}{lambda} \PY{n}{x}\PY{p}{:} \PY{n}{np}\PY{o}{.}\PY{n}{asarray}\PY{p}{(}\PY{n}{Image}\PY{o}{.}\PY{n}{open}\PY{p}{(}\PY{n}{x}\PY{p}{)}\PY{o}{.}\PY{n}{resize}\PY{p}{(}\PY{p}{(}\PY{l+m+mi}{100}\PY{p}{,}\PY{l+m+mi}{75}\PY{p}{)}\PY{p}{)}\PY{p}{)}\PY{p}{)}
         
         \PY{c+c1}{\PYZsh{} images scaled to 200x150, 1/3 of original size for CNN}
         \PY{c+c1}{\PYZsh{}metadata\PYZus{}df[\PYZsq{}image\PYZsq{}] = metadata\PYZus{}df[\PYZsq{}path\PYZsq{}].map(lambda x: np.asarray(Image.open(x).resize((200, 150))))}
         
         \PY{c+c1}{\PYZsh{} change working directory back}
         \PY{n}{os}\PY{o}{.}\PY{n}{chdir}\PY{p}{(}\PY{l+s+sa}{r}\PY{l+s+s2}{\PYZdq{}}\PY{l+s+s2}{C:}\PY{l+s+s2}{\PYZbs{}}\PY{l+s+s2}{Users}\PY{l+s+s2}{\PYZbs{}}\PY{l+s+s2}{b5034806}\PY{l+s+s2}{\PYZbs{}}\PY{l+s+s2}{Documents}\PY{l+s+s2}{\PYZbs{}}\PY{l+s+s2}{Machine\PYZhy{}Learning\PYZhy{}Extended\PYZhy{}Technical\PYZhy{}Project}\PY{l+s+s2}{\PYZbs{}}\PY{l+s+s2}{mletp}\PY{l+s+s2}{\PYZbs{}}\PY{l+s+s2}{src}\PY{l+s+s2}{\PYZdq{}}\PY{p}{)}
\end{Verbatim}


    \begin{Verbatim}[commandchars=\\\{\}]
{\color{incolor}In [{\color{incolor}88}]:} \PY{c+c1}{\PYZsh{} thanks to https://www.kaggle.com/kmader/dermatology\PYZhy{}mnist\PYZhy{}loading\PYZhy{}and\PYZhy{}processing}
         \PY{c+c1}{\PYZsh{} plot some of the images}
         \PY{n}{n\PYZus{}samples} \PY{o}{=} \PY{l+m+mi}{5}
         \PY{n}{fig}\PY{p}{,} \PY{n}{m\PYZus{}axs} \PY{o}{=} \PY{n}{plt}\PY{o}{.}\PY{n}{subplots}\PY{p}{(}\PY{l+m+mi}{7}\PY{p}{,} \PY{n}{n\PYZus{}samples}\PY{p}{,} \PY{n}{figsize} \PY{o}{=} \PY{p}{(}\PY{l+m+mi}{4}\PY{o}{*}\PY{n}{n\PYZus{}samples}\PY{p}{,} \PY{l+m+mi}{3}\PY{o}{*}\PY{l+m+mi}{7}\PY{p}{)}\PY{p}{)}
         \PY{k}{for} \PY{n}{n\PYZus{}axs}\PY{p}{,} \PY{p}{(}\PY{n}{type\PYZus{}name}\PY{p}{,} \PY{n}{type\PYZus{}rows}\PY{p}{)} \PY{o+ow}{in} \PY{n+nb}{zip}\PY{p}{(}\PY{n}{m\PYZus{}axs}\PY{p}{,} 
                                                  \PY{n}{metadata\PYZus{}df}\PY{o}{.}\PY{n}{sort\PYZus{}values}\PY{p}{(}\PY{p}{[}\PY{l+s+s1}{\PYZsq{}}\PY{l+s+s1}{cell\PYZus{}type}\PY{l+s+s1}{\PYZsq{}}\PY{p}{]}\PY{p}{)}\PY{o}{.}\PY{n}{groupby}\PY{p}{(}\PY{l+s+s1}{\PYZsq{}}\PY{l+s+s1}{cell\PYZus{}type}\PY{l+s+s1}{\PYZsq{}}\PY{p}{)}\PY{p}{)}\PY{p}{:}
             \PY{n}{n\PYZus{}axs}\PY{p}{[}\PY{l+m+mi}{0}\PY{p}{]}\PY{o}{.}\PY{n}{set\PYZus{}title}\PY{p}{(}\PY{n}{type\PYZus{}name}\PY{p}{)}
             \PY{k}{for} \PY{n}{c\PYZus{}ax}\PY{p}{,} \PY{p}{(}\PY{n}{\PYZus{}}\PY{p}{,} \PY{n}{c\PYZus{}row}\PY{p}{)} \PY{o+ow}{in} \PY{n+nb}{zip}\PY{p}{(}\PY{n}{n\PYZus{}axs}\PY{p}{,} \PY{n}{type\PYZus{}rows}\PY{o}{.}\PY{n}{sample}\PY{p}{(}\PY{n}{n\PYZus{}samples}\PY{p}{,} \PY{n}{random\PYZus{}state}\PY{o}{=}\PY{l+m+mi}{66}\PY{p}{)}\PY{o}{.}\PY{n}{iterrows}\PY{p}{(}\PY{p}{)}\PY{p}{)}\PY{p}{:}
                 \PY{n}{c\PYZus{}ax}\PY{o}{.}\PY{n}{imshow}\PY{p}{(}\PY{n}{c\PYZus{}row}\PY{p}{[}\PY{l+s+s1}{\PYZsq{}}\PY{l+s+s1}{image}\PY{l+s+s1}{\PYZsq{}}\PY{p}{]}\PY{p}{)}
                 \PY{n}{c\PYZus{}ax}\PY{o}{.}\PY{n}{axis}\PY{p}{(}\PY{l+s+s1}{\PYZsq{}}\PY{l+s+s1}{off}\PY{l+s+s1}{\PYZsq{}}\PY{p}{)}
         \PY{n}{fig}\PY{o}{.}\PY{n}{savefig}\PY{p}{(}\PY{l+s+s1}{\PYZsq{}}\PY{l+s+s1}{category\PYZus{}samples.png}\PY{l+s+s1}{\PYZsq{}}\PY{p}{,} \PY{n}{dpi}\PY{o}{=}\PY{l+m+mi}{300}\PY{p}{)}
\end{Verbatim}


    \begin{center}
    \adjustimage{max size={0.9\linewidth}{0.9\paperheight}}{output_47_0.png}
    \end{center}
    { \hspace*{\fill} \\}
    
    The predictor variables and response variable were split, so that the
model would attempt to learn what cell type was in the image i.e.~the
image ID.

    \begin{Verbatim}[commandchars=\\\{\}]
{\color{incolor}In [{\color{incolor}89}]:} \PY{c+c1}{\PYZsh{} get predictor variables and add them to a dataframe}
         \PY{n}{predictor} \PY{o}{=} \PY{n}{metadata\PYZus{}df}\PY{o}{.}\PY{n}{drop}\PY{p}{(}\PY{n}{columns}\PY{o}{=}\PY{p}{[}\PY{l+s+s1}{\PYZsq{}}\PY{l+s+s1}{cell\PYZus{}type\PYZus{}id}\PY{l+s+s1}{\PYZsq{}}\PY{p}{]}\PY{p}{,}\PY{n}{axis}\PY{o}{=}\PY{l+m+mi}{1}\PY{p}{)}
         \PY{c+c1}{\PYZsh{} get response variables and add them to a series}
         \PY{n}{response} \PY{o}{=} \PY{n}{metadata\PYZus{}df}\PY{p}{[}\PY{l+s+s1}{\PYZsq{}}\PY{l+s+s1}{cell\PYZus{}type\PYZus{}id}\PY{l+s+s1}{\PYZsq{}}\PY{p}{]}
\end{Verbatim}


    It was important to find a good balance between the training and test
split, as a split too large will result in greater performance statistic
variance, whereas a split too small will result in greater parameter
estimate variance.

    \begin{Verbatim}[commandchars=\\\{\}]
{\color{incolor}In [{\color{incolor}90}]:} \PY{c+c1}{\PYZsh{} create training and test data for x and y variables at a 75:25 ratio}
         \PY{n}{x\PYZus{}train}\PY{p}{,} \PY{n}{x\PYZus{}test}\PY{p}{,} \PY{n}{y\PYZus{}train}\PY{p}{,} \PY{n}{y\PYZus{}test} \PY{o}{=} \PY{n}{train\PYZus{}test\PYZus{}split}\PY{p}{(}\PY{n}{predictor}\PY{p}{,} \PY{n}{response}\PY{p}{,} \PY{n}{test\PYZus{}size}\PY{o}{=}\PY{l+m+mf}{0.25}\PY{p}{,} \PY{n}{random\PYZus{}state}\PY{o}{=}\PY{l+m+mi}{66}\PY{p}{)}
\end{Verbatim}


    Data was normalised to get the values on a common scale, while also not
distorting the differences in the range.

    \begin{Verbatim}[commandchars=\\\{\}]
{\color{incolor}In [{\color{incolor}91}]:} \PY{c+c1}{\PYZsh{} normalise data}
         \PY{n}{x\PYZus{}train} \PY{o}{=} \PY{n}{np}\PY{o}{.}\PY{n}{asarray}\PY{p}{(}\PY{n}{x\PYZus{}train}\PY{p}{[}\PY{l+s+s2}{\PYZdq{}}\PY{l+s+s2}{image}\PY{l+s+s2}{\PYZdq{}}\PY{p}{]}\PY{o}{.}\PY{n}{tolist}\PY{p}{(}\PY{p}{)}\PY{p}{)}
         \PY{n}{x\PYZus{}test} \PY{o}{=} \PY{n}{np}\PY{o}{.}\PY{n}{asarray}\PY{p}{(}\PY{n}{x\PYZus{}test}\PY{p}{[}\PY{l+s+s1}{\PYZsq{}}\PY{l+s+s1}{image}\PY{l+s+s1}{\PYZsq{}}\PY{p}{]}\PY{o}{.}\PY{n}{tolist}\PY{p}{(}\PY{p}{)}\PY{p}{)}
         
         \PY{n}{x\PYZus{}train} \PY{o}{=} \PY{p}{(}\PY{n}{x\PYZus{}train} \PY{o}{\PYZhy{}} \PY{n}{np}\PY{o}{.}\PY{n}{mean}\PY{p}{(}\PY{n}{x\PYZus{}train}\PY{p}{)}\PY{p}{)} \PY{o}{/} \PY{p}{(}\PY{n}{np}\PY{o}{.}\PY{n}{std}\PY{p}{(}\PY{n}{x\PYZus{}train}\PY{p}{)}\PY{p}{)}
         \PY{n}{x\PYZus{}test} \PY{o}{=} \PY{p}{(}\PY{n}{x\PYZus{}test} \PY{o}{\PYZhy{}} \PY{n}{np}\PY{o}{.}\PY{n}{mean}\PY{p}{(}\PY{n}{x\PYZus{}test}\PY{p}{)} \PY{o}{/} \PY{n}{np}\PY{o}{.}\PY{n}{std}\PY{p}{(}\PY{n}{x\PYZus{}test}\PY{p}{)}\PY{p}{)}
\end{Verbatim}


    \begin{Verbatim}[commandchars=\\\{\}]
{\color{incolor}In [{\color{incolor}92}]:} \PY{c+c1}{\PYZsh{} split data into training and validation data set at a 70:30 ratio}
         \PY{c+c1}{\PYZsh{}x\PYZus{}train, x\PYZus{}validate, y\PYZus{}train, y\PYZus{}validate = train\PYZus{}test\PYZus{}split(x\PYZus{}train, y\PYZus{}train, test\PYZus{}size = 0.3, random\PYZus{}state = 66)}
\end{Verbatim}


    One hot encoding was performed on the labels in order to allow the
models to bypass the issue of categorical values.

    \begin{Verbatim}[commandchars=\\\{\}]
{\color{incolor}In [{\color{incolor}93}]:} \PY{c+c1}{\PYZsh{} one hot encoding}
         \PY{n}{y\PYZus{}train} \PY{o}{=} \PY{n}{to\PYZus{}categorical}\PY{p}{(}\PY{n}{y\PYZus{}train}\PY{p}{,} \PY{n}{num\PYZus{}classes} \PY{o}{=} \PY{l+m+mi}{7}\PY{p}{)}
         \PY{n}{y\PYZus{}test} \PY{o}{=} \PY{n}{to\PYZus{}categorical}\PY{p}{(}\PY{n}{y\PYZus{}test}\PY{p}{,} \PY{n}{num\PYZus{}classes} \PY{o}{=} \PY{l+m+mi}{7}\PY{p}{)}
\end{Verbatim}


    A canal value of 3 was used for the convolutional layer, as previously
explained in the `Convolutional Neural Network' section.

    \begin{Verbatim}[commandchars=\\\{\}]
{\color{incolor}In [{\color{incolor}94}]:} \PY{c+c1}{\PYZsh{} canal value of 3 as images are RGB}
         \PY{n}{x\PYZus{}train} \PY{o}{=} \PY{n}{x\PYZus{}train}\PY{o}{.}\PY{n}{reshape}\PY{p}{(}\PY{n}{x\PYZus{}train}\PY{o}{.}\PY{n}{shape}\PY{p}{[}\PY{l+m+mi}{0}\PY{p}{]}\PY{p}{,} \PY{o}{*}\PY{p}{(}\PY{l+m+mi}{75}\PY{p}{,} \PY{l+m+mi}{100}\PY{p}{,} \PY{l+m+mi}{3}\PY{p}{)}\PY{p}{)}
         \PY{n}{x\PYZus{}test} \PY{o}{=} \PY{n}{x\PYZus{}test}\PY{o}{.}\PY{n}{reshape}\PY{p}{(}\PY{n}{x\PYZus{}test}\PY{o}{.}\PY{n}{shape}\PY{p}{[}\PY{l+m+mi}{0}\PY{p}{]}\PY{p}{,} \PY{o}{*}\PY{p}{(}\PY{l+m+mi}{75}\PY{p}{,} \PY{l+m+mi}{100}\PY{p}{,} \PY{l+m+mi}{3}\PY{p}{)}\PY{p}{)}
\end{Verbatim}


    \begin{Verbatim}[commandchars=\\\{\}]
{\color{incolor}In [{\color{incolor}95}]:} \PY{c+c1}{\PYZsh{} thanks to https://github.com/yuguan1/example\PYZhy{}ML\PYZhy{}code/blob/master/DL1/CNN.ipynb}
         \PY{c+c1}{\PYZsh{} set the CNN model }
         \PY{n}{input\PYZus{}shape} \PY{o}{=} \PY{p}{(}\PY{l+m+mi}{75}\PY{p}{,} \PY{l+m+mi}{100}\PY{p}{,} \PY{l+m+mi}{3}\PY{p}{)}
         \PY{n}{nClasses} \PY{o}{=} \PY{l+m+mi}{7}
         
         \PY{n}{model} \PY{o}{=} \PY{n}{Sequential}\PY{p}{(}\PY{p}{)}
         \PY{n}{model}\PY{o}{.}\PY{n}{add}\PY{p}{(}\PY{n}{Conv2D}\PY{p}{(}\PY{l+m+mi}{32}\PY{p}{,} \PY{n}{kernel\PYZus{}size}\PY{o}{=}\PY{p}{(}\PY{l+m+mi}{3}\PY{p}{,} \PY{l+m+mi}{3}\PY{p}{)}\PY{p}{,}
                          \PY{n}{activation}\PY{o}{=}\PY{l+s+s1}{\PYZsq{}}\PY{l+s+s1}{relu}\PY{l+s+s1}{\PYZsq{}}\PY{p}{,}
                          \PY{n}{input\PYZus{}shape}\PY{o}{=}\PY{p}{(}\PY{n}{input\PYZus{}shape}\PY{p}{)}\PY{p}{)}\PY{p}{)}
         \PY{n}{model}\PY{o}{.}\PY{n}{add}\PY{p}{(}\PY{n}{MaxPooling2D}\PY{p}{(}\PY{n}{pool\PYZus{}size}\PY{o}{=}\PY{p}{(}\PY{l+m+mi}{2}\PY{p}{,} \PY{l+m+mi}{2}\PY{p}{)}\PY{p}{)}\PY{p}{)}
         \PY{n}{model}\PY{o}{.}\PY{n}{add}\PY{p}{(}\PY{n}{Conv2D}\PY{p}{(}\PY{l+m+mi}{64}\PY{p}{,} \PY{p}{(}\PY{l+m+mi}{3}\PY{p}{,} \PY{l+m+mi}{3}\PY{p}{)}\PY{p}{,} \PY{n}{activation}\PY{o}{=}\PY{l+s+s1}{\PYZsq{}}\PY{l+s+s1}{relu}\PY{l+s+s1}{\PYZsq{}}\PY{p}{)}\PY{p}{)}
         \PY{n}{model}\PY{o}{.}\PY{n}{add}\PY{p}{(}\PY{n}{MaxPooling2D}\PY{p}{(}\PY{n}{pool\PYZus{}size}\PY{o}{=}\PY{p}{(}\PY{l+m+mi}{2}\PY{p}{,} \PY{l+m+mi}{2}\PY{p}{)}\PY{p}{)}\PY{p}{)}
         \PY{n}{model}\PY{o}{.}\PY{n}{add}\PY{p}{(}\PY{n}{Dropout}\PY{p}{(}\PY{l+m+mf}{0.25}\PY{p}{)}\PY{p}{)}
         \PY{n}{model}\PY{o}{.}\PY{n}{add}\PY{p}{(}\PY{n}{Flatten}\PY{p}{(}\PY{p}{)}\PY{p}{)}
         \PY{n}{model}\PY{o}{.}\PY{n}{add}\PY{p}{(}\PY{n}{Dense}\PY{p}{(}\PY{l+m+mi}{128}\PY{p}{,} \PY{n}{activation}\PY{o}{=}\PY{l+s+s1}{\PYZsq{}}\PY{l+s+s1}{relu}\PY{l+s+s1}{\PYZsq{}}\PY{p}{)}\PY{p}{)}
         \PY{n}{model}\PY{o}{.}\PY{n}{add}\PY{p}{(}\PY{n}{Dropout}\PY{p}{(}\PY{l+m+mf}{0.5}\PY{p}{)}\PY{p}{)}
         \PY{n}{model}\PY{o}{.}\PY{n}{add}\PY{p}{(}\PY{n}{Dense}\PY{p}{(}\PY{n}{nClasses}\PY{p}{,} \PY{n}{activation}\PY{o}{=}\PY{l+s+s1}{\PYZsq{}}\PY{l+s+s1}{softmax}\PY{l+s+s1}{\PYZsq{}}\PY{p}{)}\PY{p}{)}
         
         \PY{n}{model}\PY{o}{.}\PY{n}{summary}\PY{p}{(}\PY{p}{)}
\end{Verbatim}


    \begin{Verbatim}[commandchars=\\\{\}]
\_\_\_\_\_\_\_\_\_\_\_\_\_\_\_\_\_\_\_\_\_\_\_\_\_\_\_\_\_\_\_\_\_\_\_\_\_\_\_\_\_\_\_\_\_\_\_\_\_\_\_\_\_\_\_\_\_\_\_\_\_\_\_\_\_
Layer (type)                 Output Shape              Param \#   
=================================================================
conv2d\_9 (Conv2D)            (None, 73, 98, 32)        896       
\_\_\_\_\_\_\_\_\_\_\_\_\_\_\_\_\_\_\_\_\_\_\_\_\_\_\_\_\_\_\_\_\_\_\_\_\_\_\_\_\_\_\_\_\_\_\_\_\_\_\_\_\_\_\_\_\_\_\_\_\_\_\_\_\_
max\_pooling2d\_9 (MaxPooling2 (None, 36, 49, 32)        0         
\_\_\_\_\_\_\_\_\_\_\_\_\_\_\_\_\_\_\_\_\_\_\_\_\_\_\_\_\_\_\_\_\_\_\_\_\_\_\_\_\_\_\_\_\_\_\_\_\_\_\_\_\_\_\_\_\_\_\_\_\_\_\_\_\_
conv2d\_10 (Conv2D)           (None, 34, 47, 64)        18496     
\_\_\_\_\_\_\_\_\_\_\_\_\_\_\_\_\_\_\_\_\_\_\_\_\_\_\_\_\_\_\_\_\_\_\_\_\_\_\_\_\_\_\_\_\_\_\_\_\_\_\_\_\_\_\_\_\_\_\_\_\_\_\_\_\_
max\_pooling2d\_10 (MaxPooling (None, 17, 23, 64)        0         
\_\_\_\_\_\_\_\_\_\_\_\_\_\_\_\_\_\_\_\_\_\_\_\_\_\_\_\_\_\_\_\_\_\_\_\_\_\_\_\_\_\_\_\_\_\_\_\_\_\_\_\_\_\_\_\_\_\_\_\_\_\_\_\_\_
dropout\_9 (Dropout)          (None, 17, 23, 64)        0         
\_\_\_\_\_\_\_\_\_\_\_\_\_\_\_\_\_\_\_\_\_\_\_\_\_\_\_\_\_\_\_\_\_\_\_\_\_\_\_\_\_\_\_\_\_\_\_\_\_\_\_\_\_\_\_\_\_\_\_\_\_\_\_\_\_
flatten\_5 (Flatten)          (None, 25024)             0         
\_\_\_\_\_\_\_\_\_\_\_\_\_\_\_\_\_\_\_\_\_\_\_\_\_\_\_\_\_\_\_\_\_\_\_\_\_\_\_\_\_\_\_\_\_\_\_\_\_\_\_\_\_\_\_\_\_\_\_\_\_\_\_\_\_
dense\_9 (Dense)              (None, 128)               3203200   
\_\_\_\_\_\_\_\_\_\_\_\_\_\_\_\_\_\_\_\_\_\_\_\_\_\_\_\_\_\_\_\_\_\_\_\_\_\_\_\_\_\_\_\_\_\_\_\_\_\_\_\_\_\_\_\_\_\_\_\_\_\_\_\_\_
dropout\_10 (Dropout)         (None, 128)               0         
\_\_\_\_\_\_\_\_\_\_\_\_\_\_\_\_\_\_\_\_\_\_\_\_\_\_\_\_\_\_\_\_\_\_\_\_\_\_\_\_\_\_\_\_\_\_\_\_\_\_\_\_\_\_\_\_\_\_\_\_\_\_\_\_\_
dense\_10 (Dense)             (None, 7)                 903       
=================================================================
Total params: 3,223,495
Trainable params: 3,223,495
Non-trainable params: 0
\_\_\_\_\_\_\_\_\_\_\_\_\_\_\_\_\_\_\_\_\_\_\_\_\_\_\_\_\_\_\_\_\_\_\_\_\_\_\_\_\_\_\_\_\_\_\_\_\_\_\_\_\_\_\_\_\_\_\_\_\_\_\_\_\_

    \end{Verbatim}

    The adam optimiser is a generalised optimizer that works for many
models, and is particularly efficient for deep learning.

    \begin{Verbatim}[commandchars=\\\{\}]
{\color{incolor}In [{\color{incolor}96}]:} \PY{c+c1}{\PYZsh{} compile the model}
         
         \PY{n}{model}\PY{o}{.}\PY{n}{compile}\PY{p}{(}\PY{n}{optimizer} \PY{o}{=} \PY{l+s+s1}{\PYZsq{}}\PY{l+s+s1}{adam}\PY{l+s+s1}{\PYZsq{}}\PY{p}{,} \PY{n}{loss} \PY{o}{=} \PY{l+s+s2}{\PYZdq{}}\PY{l+s+s2}{categorical\PYZus{}crossentropy}\PY{l+s+s2}{\PYZdq{}}\PY{p}{,} \PY{n}{metrics}\PY{o}{=}\PY{p}{[}\PY{l+s+s2}{\PYZdq{}}\PY{l+s+s2}{accuracy}\PY{l+s+s2}{\PYZdq{}}\PY{p}{]}\PY{p}{)}
\end{Verbatim}


    A learning rate annealer was used to stablise the learning rate. This
was used as initially, the learning rate is too high which causes a high
variance in accuracy and therefore divergant results. Stabilising the
learning rate over time reduces the impact of this. It is also worth
noting that having the learning rate too slow would require many epochs
before reaching the minimum point on the gradient descent.

    \begin{Verbatim}[commandchars=\\\{\}]
{\color{incolor}In [{\color{incolor}97}]:} \PY{c+c1}{\PYZsh{} Set a learning rate annealer}
         \PY{n}{learning\PYZus{}rate\PYZus{}reduction} \PY{o}{=} \PY{n}{ReduceLROnPlateau}\PY{p}{(}\PY{n}{monitor}\PY{o}{=}\PY{l+s+s1}{\PYZsq{}}\PY{l+s+s1}{val\PYZus{}acc}\PY{l+s+s1}{\PYZsq{}}\PY{p}{,} 
                                                     \PY{n}{patience}\PY{o}{=}\PY{l+m+mi}{3}\PY{p}{,} 
                                                     \PY{n}{verbose}\PY{o}{=}\PY{l+m+mi}{1}\PY{p}{,} 
                                                     \PY{n}{factor}\PY{o}{=}\PY{l+m+mf}{0.5}\PY{p}{,} 
                                                     \PY{n}{min\PYZus{}lr}\PY{o}{=}\PY{l+m+mf}{0.00001}\PY{p}{)}
\end{Verbatim}


    Data augmentation is used to prevent overfitting. As the dataset is not
particularly large, data augmentation helps to prevent overfitting by
generating relevant images with slight variations, to provide more data
for the dataset and the variations help to encourage generalisation and
therefore reduces the likelihood of overfitting.

    \begin{Verbatim}[commandchars=\\\{\}]
{\color{incolor}In [{\color{incolor}98}]:} \PY{c+c1}{\PYZsh{} data augmentation}
         
         \PY{n}{datagen} \PY{o}{=} \PY{n}{ImageDataGenerator}\PY{p}{(}
                 \PY{n}{rotation\PYZus{}range}\PY{o}{=}\PY{l+m+mi}{30}\PY{p}{,}  \PY{c+c1}{\PYZsh{} randomly rotate images in the range (degrees, 0 to 180)}
                 \PY{n}{horizontal\PYZus{}flip}\PY{o}{=}\PY{k+kc}{True}\PY{p}{,}  \PY{c+c1}{\PYZsh{} randomly flip images}
                 \PY{n}{vertical\PYZus{}flip}\PY{o}{=}\PY{k+kc}{True}\PY{p}{)}  \PY{c+c1}{\PYZsh{} randomly flip images}
         
         \PY{n}{datagen}\PY{o}{.}\PY{n}{fit}\PY{p}{(}\PY{n}{x\PYZus{}train}\PY{p}{)}
\end{Verbatim}


    `It has been observed in practice that when using a larger batch there
is a significant degradation in the quality of the model\ldots{}'
(Keskar, Mudigere, Nocedal, Smelyanskiy and Tang, 2016). After reading
this paper, it was clear that increasing the batch size too much would
not be good, due to generalization. Too low of a batch size would result
in a low accuracy and a slow computational process. It is also important
to consider that there may be interactions with other hyperparameters.
Therefore, epochs and batch sizes were experimented with to find the
best balance between accuracy and speed.

    \begin{Verbatim}[commandchars=\\\{\}]
{\color{incolor}In [{\color{incolor}99}]:} \PY{c+c1}{\PYZsh{} train the model CNN RESULTS}
         \PY{n}{epochs} \PY{o}{=} \PY{l+m+mi}{30} 
         \PY{n}{batch\PYZus{}size} \PY{o}{=} \PY{l+m+mi}{16}
         \PY{n}{history} \PY{o}{=} \PY{n}{model}\PY{o}{.}\PY{n}{fit\PYZus{}generator}\PY{p}{(}\PY{n}{datagen}\PY{o}{.}\PY{n}{flow}\PY{p}{(}\PY{n}{x\PYZus{}train}\PY{p}{,}\PY{n}{y\PYZus{}train}\PY{p}{,} \PY{n}{batch\PYZus{}size}\PY{o}{=}\PY{n}{batch\PYZus{}size}\PY{p}{)}\PY{p}{,}
                                       \PY{n}{epochs} \PY{o}{=} \PY{n}{epochs}\PY{p}{,} \PY{n}{validation\PYZus{}data} \PY{o}{=} \PY{p}{(}\PY{n}{x\PYZus{}test}\PY{p}{,}\PY{n}{y\PYZus{}test}\PY{p}{)}\PY{p}{,}
                                       \PY{n}{verbose} \PY{o}{=} \PY{l+m+mi}{1}\PY{p}{,} \PY{n}{steps\PYZus{}per\PYZus{}epoch}\PY{o}{=}\PY{n}{x\PYZus{}train}\PY{o}{.}\PY{n}{shape}\PY{p}{[}\PY{l+m+mi}{0}\PY{p}{]} \PY{o}{/}\PY{o}{/} \PY{n}{batch\PYZus{}size}
                                       \PY{p}{,} \PY{n}{callbacks}\PY{o}{=}\PY{p}{[}\PY{n}{learning\PYZus{}rate\PYZus{}reduction}\PY{p}{]}\PY{p}{)}
\end{Verbatim}


    \begin{Verbatim}[commandchars=\\\{\}]
Epoch 1/30
469/469 [==============================] - 10s 22ms/step - loss: 0.9886 - acc: 0.6671 - val\_loss: 8.4592 - val\_acc: 0.3990s - loss: 1.1878 - - ETA:
Epoch 2/30
469/469 [==============================] - 10s 21ms/step - loss: 0.8769 - acc: 0.6903 - val\_loss: 5.2785 - val\_acc: 0.6725 ETA: 0s - loss: 0.8821 -  - ETA: 0s - loss: 0.8774 - acc: 0.
Epoch 3/30
469/469 [==============================] - 10s 21ms/step - loss: 0.8237 - acc: 0.7045 - val\_loss: 5.1420 - val\_acc: 0.5300
Epoch 4/30
469/469 [==============================] - 10s 21ms/step - loss: 0.7886 - acc: 0.7161 - val\_loss: 5.5552 - val\_acc: 0.5016
Epoch 5/30
469/469 [==============================] - 10s 21ms/step - loss: 0.7739 - acc: 0.7290 - val\_loss: 5.2847 - val\_acc: 0.6721

Epoch 00005: ReduceLROnPlateau reducing learning rate to 0.0005000000237487257.
Epoch 6/30
469/469 [==============================] - 10s 21ms/step - loss: 0.7217 - acc: 0.7357 - val\_loss: 6.3815 - val\_acc: 0.5915
Epoch 7/30
469/469 [==============================] - 10s 21ms/step - loss: 0.7011 - acc: 0.7442 - val\_loss: 5.3121 - val\_acc: 0.6685
Epoch 8/30
469/469 [==============================] - 10s 21ms/step - loss: 0.6904 - acc: 0.7504 - val\_loss: 8.5838 - val\_acc: 0.42056909 - a - ETA: 2s - loss: 0.6924 - ETA: 1s

Epoch 00008: ReduceLROnPlateau reducing learning rate to 0.0002500000118743628.
Epoch 9/30
469/469 [==============================] - 10s 21ms/step - loss: 0.6638 - acc: 0.7533 - val\_loss: 5.3217 - val\_acc: 0.6649
Epoch 10/30
469/469 [==============================] - 10s 21ms/step - loss: 0.6513 - acc: 0.7546 - val\_loss: 5.4703 - val\_acc: 0.6585
Epoch 11/30
469/469 [==============================] - 10s 21ms/step - loss: 0.6464 - acc: 0.7642 - val\_loss: 5.5988 - val\_acc: 0.6494

Epoch 00011: ReduceLROnPlateau reducing learning rate to 0.0001250000059371814.
Epoch 12/30
469/469 [==============================] - 10s 21ms/step - loss: 0.6308 - acc: 0.7632 - val\_loss: 5.3329 - val\_acc: 0.6677
Epoch 13/30
469/469 [==============================] - 10s 21ms/step - loss: 0.6203 - acc: 0.7709 - val\_loss: 5.3173 - val\_acc: 0.6693 - loss: 
Epoch 14/30
469/469 [==============================] - 10s 21ms/step - loss: 0.6209 - acc: 0.7672 - val\_loss: 5.3793 - val\_acc: 0.6637

Epoch 00014: ReduceLROnPlateau reducing learning rate to 6.25000029685907e-05.
Epoch 15/30
469/469 [==============================] - 10s 21ms/step - loss: 0.6051 - acc: 0.7700 - val\_loss: 5.3441 - val\_acc: 0.6661
Epoch 16/30
469/469 [==============================] - 10s 21ms/step - loss: 0.6184 - acc: 0.7683 - val\_loss: 5.3557 - val\_acc: 0.6657
Epoch 17/30
469/469 [==============================] - 10s 21ms/step - loss: 0.6021 - acc: 0.7724 - val\_loss: 5.3636 - val\_acc: 0.6645

Epoch 00017: ReduceLROnPlateau reducing learning rate to 3.125000148429535e-05.
Epoch 18/30
469/469 [==============================] - 10s 21ms/step - loss: 0.6001 - acc: 0.7764 - val\_loss: 5.3480 - val\_acc: 0.6669
Epoch 19/30
469/469 [==============================] - 10s 21ms/step - loss: 0.6009 - acc: 0.7742 - val\_loss: 5.3488 - val\_acc: 0.6669
Epoch 20/30
469/469 [==============================] - 10s 21ms/step - loss: 0.5999 - acc: 0.7796 - val\_loss: 5.3312 - val\_acc: 0.6677

Epoch 00020: ReduceLROnPlateau reducing learning rate to 1.5625000742147677e-05.
Epoch 21/30
469/469 [==============================] - 10s 21ms/step - loss: 0.5906 - acc: 0.7768 - val\_loss: 5.3433 - val\_acc: 0.6661
Epoch 22/30
469/469 [==============================] - 10s 21ms/step - loss: 0.5878 - acc: 0.7794 - val\_loss: 5.3707 - val\_acc: 0.6649
Epoch 23/30
469/469 [==============================] - 10s 21ms/step - loss: 0.5915 - acc: 0.7784 - val\_loss: 5.3654 - val\_acc: 0.6653

Epoch 00023: ReduceLROnPlateau reducing learning rate to 1e-05.
Epoch 24/30
469/469 [==============================] - 10s 21ms/step - loss: 0.5939 - acc: 0.7806 - val\_loss: 5.3728 - val\_acc: 0.6649
Epoch 25/30
469/469 [==============================] - 10s 21ms/step - loss: 0.5927 - acc: 0.7801 - val\_loss: 5.3644 - val\_acc: 0.6645
Epoch 26/30
469/469 [==============================] - 10s 21ms/step - loss: 0.5869 - acc: 0.7802 - val\_loss: 5.3664 - val\_acc: 0.6657
Epoch 27/30
469/469 [==============================] - 10s 21ms/step - loss: 0.5838 - acc: 0.7755 - val\_loss: 5.3661 - val\_acc: 0.6653
Epoch 28/30
469/469 [==============================] - 10s 21ms/step - loss: 0.5990 - acc: 0.7771 - val\_loss: 5.3732 - val\_acc: 0.6653
Epoch 29/30
469/469 [==============================] - 10s 21ms/step - loss: 0.5885 - acc: 0.7794 - val\_loss: 5.3747 - val\_acc: 0.6649
Epoch 30/30
469/469 [==============================] - 10s 21ms/step - loss: 0.5778 - acc: 0.7894 - val\_loss: 5.3748 - val\_acc: 0.6649

    \end{Verbatim}

    \begin{Verbatim}[commandchars=\\\{\}]
{\color{incolor}In [{\color{incolor}100}]:} \PY{c+c1}{\PYZsh{} thanks to https://www.kaggle.com/sid321axn/step\PYZhy{}wise\PYZhy{}approach\PYZhy{}cnn\PYZhy{}model\PYZhy{}77\PYZhy{}0344\PYZhy{}accuracy}
          \PY{c+c1}{\PYZsh{} function to plot model\PYZsq{}s validation loss and validation accuracy}
          \PY{k}{def} \PY{n+nf}{plot\PYZus{}model\PYZus{}history}\PY{p}{(}\PY{n}{model\PYZus{}history}\PY{p}{)}\PY{p}{:}
              \PY{n}{fig}\PY{p}{,} \PY{n}{axs} \PY{o}{=} \PY{n}{plt}\PY{o}{.}\PY{n}{subplots}\PY{p}{(}\PY{l+m+mi}{1}\PY{p}{,}\PY{l+m+mi}{2}\PY{p}{,}\PY{n}{figsize}\PY{o}{=}\PY{p}{(}\PY{l+m+mi}{15}\PY{p}{,}\PY{l+m+mi}{5}\PY{p}{)}\PY{p}{)}
              \PY{c+c1}{\PYZsh{} summarize history for accuracy}
              \PY{n}{axs}\PY{p}{[}\PY{l+m+mi}{0}\PY{p}{]}\PY{o}{.}\PY{n}{plot}\PY{p}{(}\PY{n+nb}{range}\PY{p}{(}\PY{l+m+mi}{1}\PY{p}{,}\PY{n+nb}{len}\PY{p}{(}\PY{n}{model\PYZus{}history}\PY{o}{.}\PY{n}{history}\PY{p}{[}\PY{l+s+s1}{\PYZsq{}}\PY{l+s+s1}{acc}\PY{l+s+s1}{\PYZsq{}}\PY{p}{]}\PY{p}{)}\PY{o}{+}\PY{l+m+mi}{1}\PY{p}{)}\PY{p}{,}\PY{n}{model\PYZus{}history}\PY{o}{.}\PY{n}{history}\PY{p}{[}\PY{l+s+s1}{\PYZsq{}}\PY{l+s+s1}{acc}\PY{l+s+s1}{\PYZsq{}}\PY{p}{]}\PY{p}{)}
              \PY{n}{axs}\PY{p}{[}\PY{l+m+mi}{0}\PY{p}{]}\PY{o}{.}\PY{n}{plot}\PY{p}{(}\PY{n+nb}{range}\PY{p}{(}\PY{l+m+mi}{1}\PY{p}{,}\PY{n+nb}{len}\PY{p}{(}\PY{n}{model\PYZus{}history}\PY{o}{.}\PY{n}{history}\PY{p}{[}\PY{l+s+s1}{\PYZsq{}}\PY{l+s+s1}{val\PYZus{}acc}\PY{l+s+s1}{\PYZsq{}}\PY{p}{]}\PY{p}{)}\PY{o}{+}\PY{l+m+mi}{1}\PY{p}{)}\PY{p}{,}\PY{n}{model\PYZus{}history}\PY{o}{.}\PY{n}{history}\PY{p}{[}\PY{l+s+s1}{\PYZsq{}}\PY{l+s+s1}{val\PYZus{}acc}\PY{l+s+s1}{\PYZsq{}}\PY{p}{]}\PY{p}{)}
              \PY{n}{axs}\PY{p}{[}\PY{l+m+mi}{0}\PY{p}{]}\PY{o}{.}\PY{n}{set\PYZus{}title}\PY{p}{(}\PY{l+s+s1}{\PYZsq{}}\PY{l+s+s1}{Model Accuracy}\PY{l+s+s1}{\PYZsq{}}\PY{p}{)}
              \PY{n}{axs}\PY{p}{[}\PY{l+m+mi}{0}\PY{p}{]}\PY{o}{.}\PY{n}{set\PYZus{}ylabel}\PY{p}{(}\PY{l+s+s1}{\PYZsq{}}\PY{l+s+s1}{Accuracy}\PY{l+s+s1}{\PYZsq{}}\PY{p}{)}
              \PY{n}{axs}\PY{p}{[}\PY{l+m+mi}{0}\PY{p}{]}\PY{o}{.}\PY{n}{set\PYZus{}xlabel}\PY{p}{(}\PY{l+s+s1}{\PYZsq{}}\PY{l+s+s1}{Epoch}\PY{l+s+s1}{\PYZsq{}}\PY{p}{)}
              \PY{n}{axs}\PY{p}{[}\PY{l+m+mi}{0}\PY{p}{]}\PY{o}{.}\PY{n}{set\PYZus{}xticks}\PY{p}{(}\PY{n}{np}\PY{o}{.}\PY{n}{arange}\PY{p}{(}\PY{l+m+mi}{1}\PY{p}{,}\PY{n+nb}{len}\PY{p}{(}\PY{n}{model\PYZus{}history}\PY{o}{.}\PY{n}{history}\PY{p}{[}\PY{l+s+s1}{\PYZsq{}}\PY{l+s+s1}{acc}\PY{l+s+s1}{\PYZsq{}}\PY{p}{]}\PY{p}{)}\PY{o}{+}\PY{l+m+mi}{1}\PY{p}{)}\PY{p}{,}\PY{n+nb}{len}\PY{p}{(}\PY{n}{model\PYZus{}history}\PY{o}{.}\PY{n}{history}\PY{p}{[}\PY{l+s+s1}{\PYZsq{}}\PY{l+s+s1}{acc}\PY{l+s+s1}{\PYZsq{}}\PY{p}{]}\PY{p}{)}\PY{o}{/}\PY{l+m+mi}{10}\PY{p}{)}
              \PY{n}{axs}\PY{p}{[}\PY{l+m+mi}{0}\PY{p}{]}\PY{o}{.}\PY{n}{legend}\PY{p}{(}\PY{p}{[}\PY{l+s+s1}{\PYZsq{}}\PY{l+s+s1}{train}\PY{l+s+s1}{\PYZsq{}}\PY{p}{,} \PY{l+s+s1}{\PYZsq{}}\PY{l+s+s1}{val}\PY{l+s+s1}{\PYZsq{}}\PY{p}{]}\PY{p}{,} \PY{n}{loc}\PY{o}{=}\PY{l+s+s1}{\PYZsq{}}\PY{l+s+s1}{best}\PY{l+s+s1}{\PYZsq{}}\PY{p}{)}
              \PY{c+c1}{\PYZsh{} summarize history for loss}
              \PY{n}{axs}\PY{p}{[}\PY{l+m+mi}{1}\PY{p}{]}\PY{o}{.}\PY{n}{plot}\PY{p}{(}\PY{n+nb}{range}\PY{p}{(}\PY{l+m+mi}{1}\PY{p}{,}\PY{n+nb}{len}\PY{p}{(}\PY{n}{model\PYZus{}history}\PY{o}{.}\PY{n}{history}\PY{p}{[}\PY{l+s+s1}{\PYZsq{}}\PY{l+s+s1}{loss}\PY{l+s+s1}{\PYZsq{}}\PY{p}{]}\PY{p}{)}\PY{o}{+}\PY{l+m+mi}{1}\PY{p}{)}\PY{p}{,}\PY{n}{model\PYZus{}history}\PY{o}{.}\PY{n}{history}\PY{p}{[}\PY{l+s+s1}{\PYZsq{}}\PY{l+s+s1}{loss}\PY{l+s+s1}{\PYZsq{}}\PY{p}{]}\PY{p}{)}
              \PY{n}{axs}\PY{p}{[}\PY{l+m+mi}{1}\PY{p}{]}\PY{o}{.}\PY{n}{plot}\PY{p}{(}\PY{n+nb}{range}\PY{p}{(}\PY{l+m+mi}{1}\PY{p}{,}\PY{n+nb}{len}\PY{p}{(}\PY{n}{model\PYZus{}history}\PY{o}{.}\PY{n}{history}\PY{p}{[}\PY{l+s+s1}{\PYZsq{}}\PY{l+s+s1}{val\PYZus{}loss}\PY{l+s+s1}{\PYZsq{}}\PY{p}{]}\PY{p}{)}\PY{o}{+}\PY{l+m+mi}{1}\PY{p}{)}\PY{p}{,}\PY{n}{model\PYZus{}history}\PY{o}{.}\PY{n}{history}\PY{p}{[}\PY{l+s+s1}{\PYZsq{}}\PY{l+s+s1}{val\PYZus{}loss}\PY{l+s+s1}{\PYZsq{}}\PY{p}{]}\PY{p}{)}
              \PY{n}{axs}\PY{p}{[}\PY{l+m+mi}{1}\PY{p}{]}\PY{o}{.}\PY{n}{set\PYZus{}title}\PY{p}{(}\PY{l+s+s1}{\PYZsq{}}\PY{l+s+s1}{Model Loss}\PY{l+s+s1}{\PYZsq{}}\PY{p}{)}
              \PY{n}{axs}\PY{p}{[}\PY{l+m+mi}{1}\PY{p}{]}\PY{o}{.}\PY{n}{set\PYZus{}ylabel}\PY{p}{(}\PY{l+s+s1}{\PYZsq{}}\PY{l+s+s1}{Loss}\PY{l+s+s1}{\PYZsq{}}\PY{p}{)}
              \PY{n}{axs}\PY{p}{[}\PY{l+m+mi}{1}\PY{p}{]}\PY{o}{.}\PY{n}{set\PYZus{}xlabel}\PY{p}{(}\PY{l+s+s1}{\PYZsq{}}\PY{l+s+s1}{Epoch}\PY{l+s+s1}{\PYZsq{}}\PY{p}{)}
              \PY{n}{axs}\PY{p}{[}\PY{l+m+mi}{1}\PY{p}{]}\PY{o}{.}\PY{n}{set\PYZus{}xticks}\PY{p}{(}\PY{n}{np}\PY{o}{.}\PY{n}{arange}\PY{p}{(}\PY{l+m+mi}{1}\PY{p}{,}\PY{n+nb}{len}\PY{p}{(}\PY{n}{model\PYZus{}history}\PY{o}{.}\PY{n}{history}\PY{p}{[}\PY{l+s+s1}{\PYZsq{}}\PY{l+s+s1}{loss}\PY{l+s+s1}{\PYZsq{}}\PY{p}{]}\PY{p}{)}\PY{o}{+}\PY{l+m+mi}{1}\PY{p}{)}\PY{p}{,}\PY{n+nb}{len}\PY{p}{(}\PY{n}{model\PYZus{}history}\PY{o}{.}\PY{n}{history}\PY{p}{[}\PY{l+s+s1}{\PYZsq{}}\PY{l+s+s1}{loss}\PY{l+s+s1}{\PYZsq{}}\PY{p}{]}\PY{p}{)}\PY{o}{/}\PY{l+m+mi}{10}\PY{p}{)}
              \PY{n}{axs}\PY{p}{[}\PY{l+m+mi}{1}\PY{p}{]}\PY{o}{.}\PY{n}{legend}\PY{p}{(}\PY{p}{[}\PY{l+s+s1}{\PYZsq{}}\PY{l+s+s1}{train}\PY{l+s+s1}{\PYZsq{}}\PY{p}{,} \PY{l+s+s1}{\PYZsq{}}\PY{l+s+s1}{val}\PY{l+s+s1}{\PYZsq{}}\PY{p}{]}\PY{p}{,} \PY{n}{loc}\PY{o}{=}\PY{l+s+s1}{\PYZsq{}}\PY{l+s+s1}{best}\PY{l+s+s1}{\PYZsq{}}\PY{p}{)}
              \PY{n}{plt}\PY{o}{.}\PY{n}{show}\PY{p}{(}\PY{p}{)}
\end{Verbatim}


    \begin{Verbatim}[commandchars=\\\{\}]
{\color{incolor}In [{\color{incolor}101}]:} \PY{n}{plot\PYZus{}model\PYZus{}history}\PY{p}{(}\PY{n}{history}\PY{p}{)}
\end{Verbatim}


    \begin{center}
    \adjustimage{max size={0.9\linewidth}{0.9\paperheight}}{output_69_0.png}
    \end{center}
    { \hspace*{\fill} \\}
    
    From this plot we can immediately see how the learning rate annealer
affects the model. Initially there are huge spikes in the accuracy for
the test data, although after some epochs this stablises roughly at its
peak (Approx 66\%) due to the learning rate being reduced. This also
applies to the test data's loss, which we can see stablises at the same
Epoch as the accuracy. While the accuracy score is okay, the loss value
is incredibly high and suggests that a lot of errors are being made
every epoch, which may be due to the overfitting which will be explored
with the next plot.

    \begin{Verbatim}[commandchars=\\\{\}]
{\color{incolor}In [{\color{incolor}104}]:} \PY{c+c1}{\PYZsh{} thanks to https://www.kaggle.com/sid321axn/step\PYZhy{}wise\PYZhy{}approach\PYZhy{}cnn\PYZhy{}model\PYZhy{}77\PYZhy{}0344\PYZhy{}accuracy}
          \PY{n}{y\PYZus{}pred} \PY{o}{=} \PY{n}{model}\PY{o}{.}\PY{n}{predict}\PY{p}{(}\PY{n}{x\PYZus{}test}\PY{p}{)}
          \PY{n}{y\PYZus{}pred} \PY{o}{=} \PY{n}{np}\PY{o}{.}\PY{n}{argmax}\PY{p}{(}\PY{n}{y\PYZus{}pred}\PY{p}{,} \PY{n}{axis} \PY{o}{=} \PY{l+m+mi}{1}\PY{p}{)}
          \PY{n}{y\PYZus{}true} \PY{o}{=} \PY{n}{np}\PY{o}{.}\PY{n}{argmax}\PY{p}{(}\PY{n}{y\PYZus{}test}\PY{p}{,}\PY{n}{axis} \PY{o}{=} \PY{l+m+mi}{1}\PY{p}{)} 
          
          \PY{n}{confusion} \PY{o}{=} \PY{n}{confusion\PYZus{}matrix}\PY{p}{(}\PY{n}{y\PYZus{}pred}\PY{p}{,} \PY{n}{y\PYZus{}true}\PY{p}{)}
          
          \PY{c+c1}{\PYZsh{} Function to plot confusion matrix    }
          \PY{k}{def} \PY{n+nf}{plot\PYZus{}confusion\PYZus{}matrix}\PY{p}{(}\PY{n}{cm}\PY{p}{,} \PY{n}{classes}\PY{p}{,}
                                    \PY{n}{normalize}\PY{o}{=}\PY{k+kc}{False}\PY{p}{,}
                                    \PY{n}{title}\PY{o}{=}\PY{l+s+s1}{\PYZsq{}}\PY{l+s+s1}{Confusion matrix}\PY{l+s+s1}{\PYZsq{}}\PY{p}{,}
                                    \PY{n}{cmap}\PY{o}{=}\PY{n}{plt}\PY{o}{.}\PY{n}{cm}\PY{o}{.}\PY{n}{Blues}\PY{p}{)}\PY{p}{:}
              \PY{l+s+sd}{\PYZdq{}\PYZdq{}\PYZdq{}}
          \PY{l+s+sd}{    This function prints and plots the confusion matrix.}
          \PY{l+s+sd}{    Normalization can be applied by setting `normalize=True`.}
          \PY{l+s+sd}{    \PYZdq{}\PYZdq{}\PYZdq{}}
              \PY{n}{plt}\PY{o}{.}\PY{n}{imshow}\PY{p}{(}\PY{n}{cm}\PY{p}{,} \PY{n}{interpolation}\PY{o}{=}\PY{l+s+s1}{\PYZsq{}}\PY{l+s+s1}{nearest}\PY{l+s+s1}{\PYZsq{}}\PY{p}{,} \PY{n}{cmap}\PY{o}{=}\PY{n}{cmap}\PY{p}{)}
              \PY{n}{plt}\PY{o}{.}\PY{n}{title}\PY{p}{(}\PY{n}{title}\PY{p}{)}
              \PY{n}{plt}\PY{o}{.}\PY{n}{colorbar}\PY{p}{(}\PY{p}{)}
              \PY{n}{tick\PYZus{}marks} \PY{o}{=} \PY{n}{np}\PY{o}{.}\PY{n}{arange}\PY{p}{(}\PY{n+nb}{len}\PY{p}{(}\PY{n}{classes}\PY{p}{)}\PY{p}{)}
              \PY{n}{plt}\PY{o}{.}\PY{n}{xticks}\PY{p}{(}\PY{n}{tick\PYZus{}marks}\PY{p}{,} \PY{n}{classes}\PY{p}{,} \PY{n}{rotation}\PY{o}{=}\PY{l+m+mi}{45}\PY{p}{)}
              \PY{n}{plt}\PY{o}{.}\PY{n}{yticks}\PY{p}{(}\PY{n}{tick\PYZus{}marks}\PY{p}{,} \PY{n}{classes}\PY{p}{)}
          
              \PY{k}{if} \PY{n}{normalize}\PY{p}{:}
                  \PY{n}{cm} \PY{o}{=} \PY{n}{cm}\PY{o}{.}\PY{n}{astype}\PY{p}{(}\PY{l+s+s1}{\PYZsq{}}\PY{l+s+s1}{float}\PY{l+s+s1}{\PYZsq{}}\PY{p}{)} \PY{o}{/} \PY{n}{cm}\PY{o}{.}\PY{n}{sum}\PY{p}{(}\PY{n}{axis}\PY{o}{=}\PY{l+m+mi}{1}\PY{p}{)}\PY{p}{[}\PY{p}{:}\PY{p}{,} \PY{n}{np}\PY{o}{.}\PY{n}{newaxis}\PY{p}{]}
          
              \PY{n}{thresh} \PY{o}{=} \PY{n}{cm}\PY{o}{.}\PY{n}{max}\PY{p}{(}\PY{p}{)} \PY{o}{/} \PY{l+m+mf}{2.}
              \PY{k}{for} \PY{n}{i}\PY{p}{,} \PY{n}{j} \PY{o+ow}{in} \PY{n}{itertools}\PY{o}{.}\PY{n}{product}\PY{p}{(}\PY{n+nb}{range}\PY{p}{(}\PY{n}{cm}\PY{o}{.}\PY{n}{shape}\PY{p}{[}\PY{l+m+mi}{0}\PY{p}{]}\PY{p}{)}\PY{p}{,} \PY{n+nb}{range}\PY{p}{(}\PY{n}{cm}\PY{o}{.}\PY{n}{shape}\PY{p}{[}\PY{l+m+mi}{1}\PY{p}{]}\PY{p}{)}\PY{p}{)}\PY{p}{:}
                  \PY{n}{plt}\PY{o}{.}\PY{n}{text}\PY{p}{(}\PY{n}{j}\PY{p}{,} \PY{n}{i}\PY{p}{,} \PY{n}{cm}\PY{p}{[}\PY{n}{i}\PY{p}{,} \PY{n}{j}\PY{p}{]}\PY{p}{,}
                           \PY{n}{horizontalalignment}\PY{o}{=}\PY{l+s+s2}{\PYZdq{}}\PY{l+s+s2}{center}\PY{l+s+s2}{\PYZdq{}}\PY{p}{,}
                           \PY{n}{color}\PY{o}{=}\PY{l+s+s2}{\PYZdq{}}\PY{l+s+s2}{white}\PY{l+s+s2}{\PYZdq{}} \PY{k}{if} \PY{n}{cm}\PY{p}{[}\PY{n}{i}\PY{p}{,} \PY{n}{j}\PY{p}{]} \PY{o}{\PYZgt{}} \PY{n}{thresh} \PY{k}{else} \PY{l+s+s2}{\PYZdq{}}\PY{l+s+s2}{black}\PY{l+s+s2}{\PYZdq{}}\PY{p}{)}
          
              \PY{n}{plt}\PY{o}{.}\PY{n}{tight\PYZus{}layout}\PY{p}{(}\PY{p}{)}
              \PY{n}{plt}\PY{o}{.}\PY{n}{ylabel}\PY{p}{(}\PY{l+s+s1}{\PYZsq{}}\PY{l+s+s1}{True label}\PY{l+s+s1}{\PYZsq{}}\PY{p}{)}
              \PY{n}{plt}\PY{o}{.}\PY{n}{xlabel}\PY{p}{(}\PY{l+s+s1}{\PYZsq{}}\PY{l+s+s1}{Predicted label}\PY{l+s+s1}{\PYZsq{}}\PY{p}{)}
          
          \PY{n}{plot\PYZus{}confusion\PYZus{}matrix}\PY{p}{(}\PY{n}{confusion}\PY{p}{,} \PY{n}{classes} \PY{o}{=} \PY{n+nb}{range}\PY{p}{(}\PY{l+m+mi}{7}\PY{p}{)}\PY{p}{)}
\end{Verbatim}


    \begin{Verbatim}[commandchars=\\\{\}]

        ---------------------------------------------------------------------------

        NameError                                 Traceback (most recent call last)

        <ipython-input-104-e8c4038ddc62> in <module>
         34     plt.xlabel('Predicted label')
         35 
    ---> 36 plot\_confusion\_matrix(confusion, classes = range(7))
    

        <ipython-input-104-e8c4038ddc62> in plot\_confusion\_matrix(cm, classes, normalize, title, cmap)
         25 
         26     thresh = cm.max() / 2.
    ---> 27     for i, j in itertools.product(range(cm.shape[0]), range(cm.shape[1])):
         28         plt.text(j, i, cm[i, j],
         29                  horizontalalignment="center",
    

        NameError: name 'itertools' is not defined

    \end{Verbatim}

    \begin{center}
    \adjustimage{max size={0.9\linewidth}{0.9\paperheight}}{output_71_1.png}
    \end{center}
    { \hspace*{\fill} \\}
    
    Unfortunately, the confusion matrix confirms that the model has been
overfitted, as the model is only predicted for cell id 4, Melanocytic
Nevi, which was previously discussed to potentially cause overfitting.
As there is a significantly higher amount of Melanocytic Nevi than any
other cell, the machine gets into the habit of only predicting these
values, which likely explains the high loss experienced in the model.

    VGG Neural Network

The VGG Neural Network is a pre-defined neural network which was
developed by 2 Oxford Students for a neural network competition, and
received first place for image identification and 2nd place for image
classification. It is popular for being a very efficient CNN and
therefore in theory should perform better than the custom built CNN.

    \begin{Verbatim}[commandchars=\\\{\}]
{\color{incolor}In [{\color{incolor}27}]:} \PY{c+c1}{\PYZsh{} change working directory for resizing}
         \PY{n}{os}\PY{o}{.}\PY{n}{chdir}\PY{p}{(}\PY{l+s+sa}{r}\PY{l+s+s2}{\PYZdq{}}\PY{l+s+s2}{C:}\PY{l+s+s2}{\PYZbs{}}\PY{l+s+s2}{Users}\PY{l+s+s2}{\PYZbs{}}\PY{l+s+s2}{b5034806}\PY{l+s+s2}{\PYZbs{}}\PY{l+s+s2}{Documents}\PY{l+s+s2}{\PYZbs{}}\PY{l+s+s2}{Machine\PYZhy{}Learning\PYZhy{}Extended\PYZhy{}Technical\PYZhy{}Project}\PY{l+s+s2}{\PYZbs{}}\PY{l+s+s2}{mletp}\PY{l+s+s2}{\PYZbs{}}\PY{l+s+s2}{data}\PY{l+s+s2}{\PYZdq{}}\PY{p}{)}
         
         \PY{c+c1}{\PYZsh{} images rescaled to 224x224 for VGG}
         \PY{n}{metadata\PYZus{}df}\PY{p}{[}\PY{l+s+s1}{\PYZsq{}}\PY{l+s+s1}{image}\PY{l+s+s1}{\PYZsq{}}\PY{p}{]} \PY{o}{=} \PY{n}{metadata\PYZus{}df}\PY{p}{[}\PY{l+s+s1}{\PYZsq{}}\PY{l+s+s1}{path}\PY{l+s+s1}{\PYZsq{}}\PY{p}{]}\PY{o}{.}\PY{n}{map}\PY{p}{(}\PY{k}{lambda} \PY{n}{x}\PY{p}{:} \PY{n}{np}\PY{o}{.}\PY{n}{asarray}\PY{p}{(}\PY{n}{Image}\PY{o}{.}\PY{n}{open}\PY{p}{(}\PY{n}{x}\PY{p}{)}\PY{o}{.}\PY{n}{resize}\PY{p}{(}\PY{p}{(}\PY{l+m+mi}{224}\PY{p}{,}\PY{l+m+mi}{224}\PY{p}{)}\PY{p}{)}\PY{p}{)}\PY{p}{)}
         
         \PY{c+c1}{\PYZsh{} change working directory back}
         \PY{n}{os}\PY{o}{.}\PY{n}{chdir}\PY{p}{(}\PY{l+s+sa}{r}\PY{l+s+s2}{\PYZdq{}}\PY{l+s+s2}{C:}\PY{l+s+s2}{\PYZbs{}}\PY{l+s+s2}{Users}\PY{l+s+s2}{\PYZbs{}}\PY{l+s+s2}{b5034806}\PY{l+s+s2}{\PYZbs{}}\PY{l+s+s2}{Documents}\PY{l+s+s2}{\PYZbs{}}\PY{l+s+s2}{Machine\PYZhy{}Learning\PYZhy{}Extended\PYZhy{}Technical\PYZhy{}Project}\PY{l+s+s2}{\PYZbs{}}\PY{l+s+s2}{mletp}\PY{l+s+s2}{\PYZbs{}}\PY{l+s+s2}{src}\PY{l+s+s2}{\PYZdq{}}\PY{p}{)}
\end{Verbatim}


    \begin{Verbatim}[commandchars=\\\{\}]
{\color{incolor}In [{\color{incolor}28}]:} \PY{c+c1}{\PYZsh{} get predictor variables and add them to a dataframe}
         \PY{n}{predictor} \PY{o}{=} \PY{n}{metadata\PYZus{}df}\PY{o}{.}\PY{n}{drop}\PY{p}{(}\PY{n}{columns}\PY{o}{=}\PY{p}{[}\PY{l+s+s1}{\PYZsq{}}\PY{l+s+s1}{cell\PYZus{}type\PYZus{}id}\PY{l+s+s1}{\PYZsq{}}\PY{p}{]}\PY{p}{,}\PY{n}{axis}\PY{o}{=}\PY{l+m+mi}{1}\PY{p}{)}
         \PY{c+c1}{\PYZsh{} get response variables and add them to a series}
         \PY{n}{response} \PY{o}{=} \PY{n}{metadata\PYZus{}df}\PY{p}{[}\PY{l+s+s1}{\PYZsq{}}\PY{l+s+s1}{cell\PYZus{}type\PYZus{}id}\PY{l+s+s1}{\PYZsq{}}\PY{p}{]}
\end{Verbatim}


    \begin{Verbatim}[commandchars=\\\{\}]
{\color{incolor}In [{\color{incolor}29}]:} \PY{c+c1}{\PYZsh{} create training and test data for x and y variables at a 75:25 ratio}
         \PY{n}{x\PYZus{}train}\PY{p}{,} \PY{n}{x\PYZus{}test}\PY{p}{,} \PY{n}{y\PYZus{}train}\PY{p}{,} \PY{n}{y\PYZus{}test} \PY{o}{=} \PY{n}{train\PYZus{}test\PYZus{}split}\PY{p}{(}\PY{n}{predictor}\PY{p}{,} \PY{n}{response}\PY{p}{,} \PY{n}{test\PYZus{}size}\PY{o}{=}\PY{l+m+mf}{0.25}\PY{p}{,} \PY{n}{random\PYZus{}state}\PY{o}{=}\PY{l+m+mi}{66}\PY{p}{)}
\end{Verbatim}


    \begin{Verbatim}[commandchars=\\\{\}]
{\color{incolor}In [{\color{incolor}30}]:} \PY{c+c1}{\PYZsh{} normalise data}
         \PY{n}{x\PYZus{}train} \PY{o}{=} \PY{n}{np}\PY{o}{.}\PY{n}{asarray}\PY{p}{(}\PY{n}{x\PYZus{}train}\PY{p}{[}\PY{l+s+s2}{\PYZdq{}}\PY{l+s+s2}{image}\PY{l+s+s2}{\PYZdq{}}\PY{p}{]}\PY{o}{.}\PY{n}{tolist}\PY{p}{(}\PY{p}{)}\PY{p}{)}
         \PY{n}{x\PYZus{}test} \PY{o}{=} \PY{n}{np}\PY{o}{.}\PY{n}{asarray}\PY{p}{(}\PY{n}{x\PYZus{}test}\PY{p}{[}\PY{l+s+s1}{\PYZsq{}}\PY{l+s+s1}{image}\PY{l+s+s1}{\PYZsq{}}\PY{p}{]}\PY{o}{.}\PY{n}{tolist}\PY{p}{(}\PY{p}{)}\PY{p}{)}
         
         \PY{n}{x\PYZus{}train} \PY{o}{=} \PY{p}{(}\PY{n}{x\PYZus{}train} \PY{o}{\PYZhy{}} \PY{n}{np}\PY{o}{.}\PY{n}{mean}\PY{p}{(}\PY{n}{x\PYZus{}train}\PY{p}{)}\PY{p}{)} \PY{o}{/} \PY{p}{(}\PY{n}{np}\PY{o}{.}\PY{n}{std}\PY{p}{(}\PY{n}{x\PYZus{}train}\PY{p}{)}\PY{p}{)}
         \PY{n}{x\PYZus{}test} \PY{o}{=} \PY{p}{(}\PY{n}{x\PYZus{}test} \PY{o}{\PYZhy{}} \PY{n}{np}\PY{o}{.}\PY{n}{mean}\PY{p}{(}\PY{n}{x\PYZus{}test}\PY{p}{)} \PY{o}{/} \PY{n}{np}\PY{o}{.}\PY{n}{std}\PY{p}{(}\PY{n}{x\PYZus{}test}\PY{p}{)}\PY{p}{)}
\end{Verbatim}


    \begin{Verbatim}[commandchars=\\\{\}]
{\color{incolor}In [{\color{incolor}31}]:} \PY{c+c1}{\PYZsh{} Perform one\PYZhy{}hot encoding on the labels}
         \PY{n}{y\PYZus{}train} \PY{o}{=} \PY{n}{to\PYZus{}categorical}\PY{p}{(}\PY{n}{y\PYZus{}train}\PY{p}{,} \PY{n}{num\PYZus{}classes} \PY{o}{=} \PY{l+m+mi}{7}\PY{p}{)}
         \PY{n}{y\PYZus{}test} \PY{o}{=} \PY{n}{to\PYZus{}categorical}\PY{p}{(}\PY{n}{y\PYZus{}test}\PY{p}{,} \PY{n}{num\PYZus{}classes} \PY{o}{=} \PY{l+m+mi}{7}\PY{p}{)}
\end{Verbatim}


    \begin{Verbatim}[commandchars=\\\{\}]
{\color{incolor}In [{\color{incolor}32}]:} \PY{c+c1}{\PYZsh{} reshape images for VGG}
         \PY{n}{x\PYZus{}train} \PY{o}{=} \PY{n}{x\PYZus{}train}\PY{o}{.}\PY{n}{reshape}\PY{p}{(}\PY{n}{x\PYZus{}train}\PY{o}{.}\PY{n}{shape}\PY{p}{[}\PY{l+m+mi}{0}\PY{p}{]}\PY{p}{,} \PY{o}{*}\PY{p}{(}\PY{l+m+mi}{224}\PY{p}{,} \PY{l+m+mi}{224}\PY{p}{,} \PY{l+m+mi}{3}\PY{p}{)}\PY{p}{)}
         \PY{n}{x\PYZus{}test} \PY{o}{=} \PY{n}{x\PYZus{}test}\PY{o}{.}\PY{n}{reshape}\PY{p}{(}\PY{n}{x\PYZus{}test}\PY{o}{.}\PY{n}{shape}\PY{p}{[}\PY{l+m+mi}{0}\PY{p}{]}\PY{p}{,} \PY{o}{*}\PY{p}{(}\PY{l+m+mi}{224}\PY{p}{,} \PY{l+m+mi}{224}\PY{p}{,} \PY{l+m+mi}{3}\PY{p}{)}\PY{p}{)}
\end{Verbatim}


    \begin{Verbatim}[commandchars=\\\{\}]
{\color{incolor}In [{\color{incolor}33}]:} \PY{c+c1}{\PYZsh{} thanks to https://engmrk.com/vgg16\PYZhy{}implementation\PYZhy{}using\PYZhy{}keras/}
         \PY{c+c1}{\PYZsh{} set the VGG model}
         \PY{n}{input\PYZus{}shape} \PY{o}{=} \PY{p}{(}\PY{l+m+mi}{224}\PY{p}{,} \PY{l+m+mi}{224}\PY{p}{,} \PY{l+m+mi}{3}\PY{p}{)}
         
         \PY{n}{model} \PY{o}{=} \PY{n}{Sequential}\PY{p}{(}\PY{p}{[}
             \PY{n}{Conv2D}\PY{p}{(}\PY{l+m+mi}{64}\PY{p}{,} \PY{p}{(}\PY{l+m+mi}{3}\PY{p}{,} \PY{l+m+mi}{3}\PY{p}{)}\PY{p}{,} \PY{n}{input\PYZus{}shape}\PY{o}{=}\PY{n}{input\PYZus{}shape}\PY{p}{,} \PY{n}{padding}\PY{o}{=}\PY{l+s+s1}{\PYZsq{}}\PY{l+s+s1}{same}\PY{l+s+s1}{\PYZsq{}}\PY{p}{,}
                    \PY{n}{activation}\PY{o}{=}\PY{l+s+s1}{\PYZsq{}}\PY{l+s+s1}{relu}\PY{l+s+s1}{\PYZsq{}}\PY{p}{)}\PY{p}{,}
             \PY{n}{Conv2D}\PY{p}{(}\PY{l+m+mi}{64}\PY{p}{,} \PY{p}{(}\PY{l+m+mi}{3}\PY{p}{,} \PY{l+m+mi}{3}\PY{p}{)}\PY{p}{,} \PY{n}{activation}\PY{o}{=}\PY{l+s+s1}{\PYZsq{}}\PY{l+s+s1}{relu}\PY{l+s+s1}{\PYZsq{}}\PY{p}{,} \PY{n}{padding}\PY{o}{=}\PY{l+s+s1}{\PYZsq{}}\PY{l+s+s1}{same}\PY{l+s+s1}{\PYZsq{}}\PY{p}{)}\PY{p}{,}
             \PY{n}{MaxPooling2D}\PY{p}{(}\PY{n}{pool\PYZus{}size}\PY{o}{=}\PY{p}{(}\PY{l+m+mi}{2}\PY{p}{,} \PY{l+m+mi}{2}\PY{p}{)}\PY{p}{,} \PY{n}{strides}\PY{o}{=}\PY{p}{(}\PY{l+m+mi}{2}\PY{p}{,} \PY{l+m+mi}{2}\PY{p}{)}\PY{p}{)}\PY{p}{,}
             \PY{n}{Conv2D}\PY{p}{(}\PY{l+m+mi}{128}\PY{p}{,} \PY{p}{(}\PY{l+m+mi}{3}\PY{p}{,} \PY{l+m+mi}{3}\PY{p}{)}\PY{p}{,} \PY{n}{activation}\PY{o}{=}\PY{l+s+s1}{\PYZsq{}}\PY{l+s+s1}{relu}\PY{l+s+s1}{\PYZsq{}}\PY{p}{,} \PY{n}{padding}\PY{o}{=}\PY{l+s+s1}{\PYZsq{}}\PY{l+s+s1}{same}\PY{l+s+s1}{\PYZsq{}}\PY{p}{)}\PY{p}{,}
             \PY{n}{Conv2D}\PY{p}{(}\PY{l+m+mi}{128}\PY{p}{,} \PY{p}{(}\PY{l+m+mi}{3}\PY{p}{,} \PY{l+m+mi}{3}\PY{p}{)}\PY{p}{,} \PY{n}{activation}\PY{o}{=}\PY{l+s+s1}{\PYZsq{}}\PY{l+s+s1}{relu}\PY{l+s+s1}{\PYZsq{}}\PY{p}{,} \PY{n}{padding}\PY{o}{=}\PY{l+s+s1}{\PYZsq{}}\PY{l+s+s1}{same}\PY{l+s+s1}{\PYZsq{}}\PY{p}{,}\PY{p}{)}\PY{p}{,}
             \PY{n}{MaxPooling2D}\PY{p}{(}\PY{n}{pool\PYZus{}size}\PY{o}{=}\PY{p}{(}\PY{l+m+mi}{2}\PY{p}{,} \PY{l+m+mi}{2}\PY{p}{)}\PY{p}{,} \PY{n}{strides}\PY{o}{=}\PY{p}{(}\PY{l+m+mi}{2}\PY{p}{,} \PY{l+m+mi}{2}\PY{p}{)}\PY{p}{)}\PY{p}{,}
             \PY{n}{Conv2D}\PY{p}{(}\PY{l+m+mi}{256}\PY{p}{,} \PY{p}{(}\PY{l+m+mi}{3}\PY{p}{,} \PY{l+m+mi}{3}\PY{p}{)}\PY{p}{,} \PY{n}{activation}\PY{o}{=}\PY{l+s+s1}{\PYZsq{}}\PY{l+s+s1}{relu}\PY{l+s+s1}{\PYZsq{}}\PY{p}{,} \PY{n}{padding}\PY{o}{=}\PY{l+s+s1}{\PYZsq{}}\PY{l+s+s1}{same}\PY{l+s+s1}{\PYZsq{}}\PY{p}{,}\PY{p}{)}\PY{p}{,}
             \PY{n}{Conv2D}\PY{p}{(}\PY{l+m+mi}{256}\PY{p}{,} \PY{p}{(}\PY{l+m+mi}{3}\PY{p}{,} \PY{l+m+mi}{3}\PY{p}{)}\PY{p}{,} \PY{n}{activation}\PY{o}{=}\PY{l+s+s1}{\PYZsq{}}\PY{l+s+s1}{relu}\PY{l+s+s1}{\PYZsq{}}\PY{p}{,} \PY{n}{padding}\PY{o}{=}\PY{l+s+s1}{\PYZsq{}}\PY{l+s+s1}{same}\PY{l+s+s1}{\PYZsq{}}\PY{p}{,}\PY{p}{)}\PY{p}{,}
             \PY{n}{Conv2D}\PY{p}{(}\PY{l+m+mi}{256}\PY{p}{,} \PY{p}{(}\PY{l+m+mi}{3}\PY{p}{,} \PY{l+m+mi}{3}\PY{p}{)}\PY{p}{,} \PY{n}{activation}\PY{o}{=}\PY{l+s+s1}{\PYZsq{}}\PY{l+s+s1}{relu}\PY{l+s+s1}{\PYZsq{}}\PY{p}{,} \PY{n}{padding}\PY{o}{=}\PY{l+s+s1}{\PYZsq{}}\PY{l+s+s1}{same}\PY{l+s+s1}{\PYZsq{}}\PY{p}{,}\PY{p}{)}\PY{p}{,}
             \PY{n}{MaxPooling2D}\PY{p}{(}\PY{n}{pool\PYZus{}size}\PY{o}{=}\PY{p}{(}\PY{l+m+mi}{2}\PY{p}{,} \PY{l+m+mi}{2}\PY{p}{)}\PY{p}{,} \PY{n}{strides}\PY{o}{=}\PY{p}{(}\PY{l+m+mi}{2}\PY{p}{,} \PY{l+m+mi}{2}\PY{p}{)}\PY{p}{)}\PY{p}{,}
             \PY{n}{Conv2D}\PY{p}{(}\PY{l+m+mi}{512}\PY{p}{,} \PY{p}{(}\PY{l+m+mi}{3}\PY{p}{,} \PY{l+m+mi}{3}\PY{p}{)}\PY{p}{,} \PY{n}{activation}\PY{o}{=}\PY{l+s+s1}{\PYZsq{}}\PY{l+s+s1}{relu}\PY{l+s+s1}{\PYZsq{}}\PY{p}{,} \PY{n}{padding}\PY{o}{=}\PY{l+s+s1}{\PYZsq{}}\PY{l+s+s1}{same}\PY{l+s+s1}{\PYZsq{}}\PY{p}{,}\PY{p}{)}\PY{p}{,}
             \PY{n}{Conv2D}\PY{p}{(}\PY{l+m+mi}{512}\PY{p}{,} \PY{p}{(}\PY{l+m+mi}{3}\PY{p}{,} \PY{l+m+mi}{3}\PY{p}{)}\PY{p}{,} \PY{n}{activation}\PY{o}{=}\PY{l+s+s1}{\PYZsq{}}\PY{l+s+s1}{relu}\PY{l+s+s1}{\PYZsq{}}\PY{p}{,} \PY{n}{padding}\PY{o}{=}\PY{l+s+s1}{\PYZsq{}}\PY{l+s+s1}{same}\PY{l+s+s1}{\PYZsq{}}\PY{p}{,}\PY{p}{)}\PY{p}{,}
             \PY{n}{Conv2D}\PY{p}{(}\PY{l+m+mi}{512}\PY{p}{,} \PY{p}{(}\PY{l+m+mi}{3}\PY{p}{,} \PY{l+m+mi}{3}\PY{p}{)}\PY{p}{,} \PY{n}{activation}\PY{o}{=}\PY{l+s+s1}{\PYZsq{}}\PY{l+s+s1}{relu}\PY{l+s+s1}{\PYZsq{}}\PY{p}{,} \PY{n}{padding}\PY{o}{=}\PY{l+s+s1}{\PYZsq{}}\PY{l+s+s1}{same}\PY{l+s+s1}{\PYZsq{}}\PY{p}{,}\PY{p}{)}\PY{p}{,}
             \PY{n}{MaxPooling2D}\PY{p}{(}\PY{n}{pool\PYZus{}size}\PY{o}{=}\PY{p}{(}\PY{l+m+mi}{2}\PY{p}{,} \PY{l+m+mi}{2}\PY{p}{)}\PY{p}{,} \PY{n}{strides}\PY{o}{=}\PY{p}{(}\PY{l+m+mi}{2}\PY{p}{,} \PY{l+m+mi}{2}\PY{p}{)}\PY{p}{)}\PY{p}{,}
             \PY{n}{Conv2D}\PY{p}{(}\PY{l+m+mi}{512}\PY{p}{,} \PY{p}{(}\PY{l+m+mi}{3}\PY{p}{,} \PY{l+m+mi}{3}\PY{p}{)}\PY{p}{,} \PY{n}{activation}\PY{o}{=}\PY{l+s+s1}{\PYZsq{}}\PY{l+s+s1}{relu}\PY{l+s+s1}{\PYZsq{}}\PY{p}{,} \PY{n}{padding}\PY{o}{=}\PY{l+s+s1}{\PYZsq{}}\PY{l+s+s1}{same}\PY{l+s+s1}{\PYZsq{}}\PY{p}{,}\PY{p}{)}\PY{p}{,}
             \PY{n}{Conv2D}\PY{p}{(}\PY{l+m+mi}{512}\PY{p}{,} \PY{p}{(}\PY{l+m+mi}{3}\PY{p}{,} \PY{l+m+mi}{3}\PY{p}{)}\PY{p}{,} \PY{n}{activation}\PY{o}{=}\PY{l+s+s1}{\PYZsq{}}\PY{l+s+s1}{relu}\PY{l+s+s1}{\PYZsq{}}\PY{p}{,} \PY{n}{padding}\PY{o}{=}\PY{l+s+s1}{\PYZsq{}}\PY{l+s+s1}{same}\PY{l+s+s1}{\PYZsq{}}\PY{p}{,}\PY{p}{)}\PY{p}{,}
             \PY{n}{Conv2D}\PY{p}{(}\PY{l+m+mi}{512}\PY{p}{,} \PY{p}{(}\PY{l+m+mi}{3}\PY{p}{,} \PY{l+m+mi}{3}\PY{p}{)}\PY{p}{,} \PY{n}{activation}\PY{o}{=}\PY{l+s+s1}{\PYZsq{}}\PY{l+s+s1}{relu}\PY{l+s+s1}{\PYZsq{}}\PY{p}{,} \PY{n}{padding}\PY{o}{=}\PY{l+s+s1}{\PYZsq{}}\PY{l+s+s1}{same}\PY{l+s+s1}{\PYZsq{}}\PY{p}{,}\PY{p}{)}\PY{p}{,}
             \PY{n}{MaxPooling2D}\PY{p}{(}\PY{n}{pool\PYZus{}size}\PY{o}{=}\PY{p}{(}\PY{l+m+mi}{2}\PY{p}{,} \PY{l+m+mi}{2}\PY{p}{)}\PY{p}{,} \PY{n}{strides}\PY{o}{=}\PY{p}{(}\PY{l+m+mi}{2}\PY{p}{,} \PY{l+m+mi}{2}\PY{p}{)}\PY{p}{)}\PY{p}{,}
             \PY{n}{Flatten}\PY{p}{(}\PY{p}{)}\PY{p}{,}
             \PY{n}{Dense}\PY{p}{(}\PY{l+m+mi}{4096}\PY{p}{,} \PY{n}{activation}\PY{o}{=}\PY{l+s+s1}{\PYZsq{}}\PY{l+s+s1}{relu}\PY{l+s+s1}{\PYZsq{}}\PY{p}{)}\PY{p}{,}
             \PY{n}{Dense}\PY{p}{(}\PY{l+m+mi}{4096}\PY{p}{,} \PY{n}{activation}\PY{o}{=}\PY{l+s+s1}{\PYZsq{}}\PY{l+s+s1}{relu}\PY{l+s+s1}{\PYZsq{}}\PY{p}{)}\PY{p}{,}
             \PY{n}{Dense}\PY{p}{(}\PY{l+m+mi}{7}\PY{p}{,} \PY{n}{activation}\PY{o}{=}\PY{l+s+s1}{\PYZsq{}}\PY{l+s+s1}{softmax}\PY{l+s+s1}{\PYZsq{}}\PY{p}{)}
         \PY{p}{]}\PY{p}{)}
         
         \PY{n}{model}\PY{o}{.}\PY{n}{summary}\PY{p}{(}\PY{p}{)}
\end{Verbatim}


    \begin{Verbatim}[commandchars=\\\{\}]
\_\_\_\_\_\_\_\_\_\_\_\_\_\_\_\_\_\_\_\_\_\_\_\_\_\_\_\_\_\_\_\_\_\_\_\_\_\_\_\_\_\_\_\_\_\_\_\_\_\_\_\_\_\_\_\_\_\_\_\_\_\_\_\_\_
Layer (type)                 Output Shape              Param \#   
=================================================================
conv2d\_1 (Conv2D)            (None, 224, 224, 64)      1792      
\_\_\_\_\_\_\_\_\_\_\_\_\_\_\_\_\_\_\_\_\_\_\_\_\_\_\_\_\_\_\_\_\_\_\_\_\_\_\_\_\_\_\_\_\_\_\_\_\_\_\_\_\_\_\_\_\_\_\_\_\_\_\_\_\_
conv2d\_2 (Conv2D)            (None, 224, 224, 64)      36928     
\_\_\_\_\_\_\_\_\_\_\_\_\_\_\_\_\_\_\_\_\_\_\_\_\_\_\_\_\_\_\_\_\_\_\_\_\_\_\_\_\_\_\_\_\_\_\_\_\_\_\_\_\_\_\_\_\_\_\_\_\_\_\_\_\_
max\_pooling2d\_1 (MaxPooling2 (None, 112, 112, 64)      0         
\_\_\_\_\_\_\_\_\_\_\_\_\_\_\_\_\_\_\_\_\_\_\_\_\_\_\_\_\_\_\_\_\_\_\_\_\_\_\_\_\_\_\_\_\_\_\_\_\_\_\_\_\_\_\_\_\_\_\_\_\_\_\_\_\_
conv2d\_3 (Conv2D)            (None, 112, 112, 128)     73856     
\_\_\_\_\_\_\_\_\_\_\_\_\_\_\_\_\_\_\_\_\_\_\_\_\_\_\_\_\_\_\_\_\_\_\_\_\_\_\_\_\_\_\_\_\_\_\_\_\_\_\_\_\_\_\_\_\_\_\_\_\_\_\_\_\_
conv2d\_4 (Conv2D)            (None, 112, 112, 128)     147584    
\_\_\_\_\_\_\_\_\_\_\_\_\_\_\_\_\_\_\_\_\_\_\_\_\_\_\_\_\_\_\_\_\_\_\_\_\_\_\_\_\_\_\_\_\_\_\_\_\_\_\_\_\_\_\_\_\_\_\_\_\_\_\_\_\_
max\_pooling2d\_2 (MaxPooling2 (None, 56, 56, 128)       0         
\_\_\_\_\_\_\_\_\_\_\_\_\_\_\_\_\_\_\_\_\_\_\_\_\_\_\_\_\_\_\_\_\_\_\_\_\_\_\_\_\_\_\_\_\_\_\_\_\_\_\_\_\_\_\_\_\_\_\_\_\_\_\_\_\_
conv2d\_5 (Conv2D)            (None, 56, 56, 256)       295168    
\_\_\_\_\_\_\_\_\_\_\_\_\_\_\_\_\_\_\_\_\_\_\_\_\_\_\_\_\_\_\_\_\_\_\_\_\_\_\_\_\_\_\_\_\_\_\_\_\_\_\_\_\_\_\_\_\_\_\_\_\_\_\_\_\_
conv2d\_6 (Conv2D)            (None, 56, 56, 256)       590080    
\_\_\_\_\_\_\_\_\_\_\_\_\_\_\_\_\_\_\_\_\_\_\_\_\_\_\_\_\_\_\_\_\_\_\_\_\_\_\_\_\_\_\_\_\_\_\_\_\_\_\_\_\_\_\_\_\_\_\_\_\_\_\_\_\_
conv2d\_7 (Conv2D)            (None, 56, 56, 256)       590080    
\_\_\_\_\_\_\_\_\_\_\_\_\_\_\_\_\_\_\_\_\_\_\_\_\_\_\_\_\_\_\_\_\_\_\_\_\_\_\_\_\_\_\_\_\_\_\_\_\_\_\_\_\_\_\_\_\_\_\_\_\_\_\_\_\_
max\_pooling2d\_3 (MaxPooling2 (None, 28, 28, 256)       0         
\_\_\_\_\_\_\_\_\_\_\_\_\_\_\_\_\_\_\_\_\_\_\_\_\_\_\_\_\_\_\_\_\_\_\_\_\_\_\_\_\_\_\_\_\_\_\_\_\_\_\_\_\_\_\_\_\_\_\_\_\_\_\_\_\_
conv2d\_8 (Conv2D)            (None, 28, 28, 512)       1180160   
\_\_\_\_\_\_\_\_\_\_\_\_\_\_\_\_\_\_\_\_\_\_\_\_\_\_\_\_\_\_\_\_\_\_\_\_\_\_\_\_\_\_\_\_\_\_\_\_\_\_\_\_\_\_\_\_\_\_\_\_\_\_\_\_\_
conv2d\_9 (Conv2D)            (None, 28, 28, 512)       2359808   
\_\_\_\_\_\_\_\_\_\_\_\_\_\_\_\_\_\_\_\_\_\_\_\_\_\_\_\_\_\_\_\_\_\_\_\_\_\_\_\_\_\_\_\_\_\_\_\_\_\_\_\_\_\_\_\_\_\_\_\_\_\_\_\_\_
conv2d\_10 (Conv2D)           (None, 28, 28, 512)       2359808   
\_\_\_\_\_\_\_\_\_\_\_\_\_\_\_\_\_\_\_\_\_\_\_\_\_\_\_\_\_\_\_\_\_\_\_\_\_\_\_\_\_\_\_\_\_\_\_\_\_\_\_\_\_\_\_\_\_\_\_\_\_\_\_\_\_
max\_pooling2d\_4 (MaxPooling2 (None, 14, 14, 512)       0         
\_\_\_\_\_\_\_\_\_\_\_\_\_\_\_\_\_\_\_\_\_\_\_\_\_\_\_\_\_\_\_\_\_\_\_\_\_\_\_\_\_\_\_\_\_\_\_\_\_\_\_\_\_\_\_\_\_\_\_\_\_\_\_\_\_
conv2d\_11 (Conv2D)           (None, 14, 14, 512)       2359808   
\_\_\_\_\_\_\_\_\_\_\_\_\_\_\_\_\_\_\_\_\_\_\_\_\_\_\_\_\_\_\_\_\_\_\_\_\_\_\_\_\_\_\_\_\_\_\_\_\_\_\_\_\_\_\_\_\_\_\_\_\_\_\_\_\_
conv2d\_12 (Conv2D)           (None, 14, 14, 512)       2359808   
\_\_\_\_\_\_\_\_\_\_\_\_\_\_\_\_\_\_\_\_\_\_\_\_\_\_\_\_\_\_\_\_\_\_\_\_\_\_\_\_\_\_\_\_\_\_\_\_\_\_\_\_\_\_\_\_\_\_\_\_\_\_\_\_\_
conv2d\_13 (Conv2D)           (None, 14, 14, 512)       2359808   
\_\_\_\_\_\_\_\_\_\_\_\_\_\_\_\_\_\_\_\_\_\_\_\_\_\_\_\_\_\_\_\_\_\_\_\_\_\_\_\_\_\_\_\_\_\_\_\_\_\_\_\_\_\_\_\_\_\_\_\_\_\_\_\_\_
max\_pooling2d\_5 (MaxPooling2 (None, 7, 7, 512)         0         
\_\_\_\_\_\_\_\_\_\_\_\_\_\_\_\_\_\_\_\_\_\_\_\_\_\_\_\_\_\_\_\_\_\_\_\_\_\_\_\_\_\_\_\_\_\_\_\_\_\_\_\_\_\_\_\_\_\_\_\_\_\_\_\_\_
flatten\_1 (Flatten)          (None, 25088)             0         
\_\_\_\_\_\_\_\_\_\_\_\_\_\_\_\_\_\_\_\_\_\_\_\_\_\_\_\_\_\_\_\_\_\_\_\_\_\_\_\_\_\_\_\_\_\_\_\_\_\_\_\_\_\_\_\_\_\_\_\_\_\_\_\_\_
dense\_1 (Dense)              (None, 4096)              102764544 
\_\_\_\_\_\_\_\_\_\_\_\_\_\_\_\_\_\_\_\_\_\_\_\_\_\_\_\_\_\_\_\_\_\_\_\_\_\_\_\_\_\_\_\_\_\_\_\_\_\_\_\_\_\_\_\_\_\_\_\_\_\_\_\_\_
dense\_2 (Dense)              (None, 4096)              16781312  
\_\_\_\_\_\_\_\_\_\_\_\_\_\_\_\_\_\_\_\_\_\_\_\_\_\_\_\_\_\_\_\_\_\_\_\_\_\_\_\_\_\_\_\_\_\_\_\_\_\_\_\_\_\_\_\_\_\_\_\_\_\_\_\_\_
dense\_3 (Dense)              (None, 7)                 28679     
=================================================================
Total params: 134,289,223
Trainable params: 134,289,223
Non-trainable params: 0
\_\_\_\_\_\_\_\_\_\_\_\_\_\_\_\_\_\_\_\_\_\_\_\_\_\_\_\_\_\_\_\_\_\_\_\_\_\_\_\_\_\_\_\_\_\_\_\_\_\_\_\_\_\_\_\_\_\_\_\_\_\_\_\_\_

    \end{Verbatim}

    \begin{Verbatim}[commandchars=\\\{\}]
{\color{incolor}In [{\color{incolor}34}]:} \PY{c+c1}{\PYZsh{} compile the model}
         
         \PY{n}{model}\PY{o}{.}\PY{n}{compile}\PY{p}{(}\PY{n}{optimizer} \PY{o}{=} \PY{l+s+s1}{\PYZsq{}}\PY{l+s+s1}{adam}\PY{l+s+s1}{\PYZsq{}}\PY{p}{,} \PY{n}{loss} \PY{o}{=} \PY{l+s+s2}{\PYZdq{}}\PY{l+s+s2}{categorical\PYZus{}crossentropy}\PY{l+s+s2}{\PYZdq{}}\PY{p}{,} \PY{n}{metrics}\PY{o}{=}\PY{p}{[}\PY{l+s+s2}{\PYZdq{}}\PY{l+s+s2}{accuracy}\PY{l+s+s2}{\PYZdq{}}\PY{p}{]}\PY{p}{)}
\end{Verbatim}


    \begin{Verbatim}[commandchars=\\\{\}]
{\color{incolor}In [{\color{incolor}35}]:} \PY{c+c1}{\PYZsh{} Set a learning rate annealer}
         \PY{n}{learning\PYZus{}rate\PYZus{}reduction} \PY{o}{=} \PY{n}{ReduceLROnPlateau}\PY{p}{(}\PY{n}{monitor}\PY{o}{=}\PY{l+s+s1}{\PYZsq{}}\PY{l+s+s1}{val\PYZus{}acc}\PY{l+s+s1}{\PYZsq{}}\PY{p}{,} 
                                                     \PY{n}{patience}\PY{o}{=}\PY{l+m+mi}{3}\PY{p}{,} 
                                                     \PY{n}{verbose}\PY{o}{=}\PY{l+m+mi}{1}\PY{p}{,} 
                                                     \PY{n}{factor}\PY{o}{=}\PY{l+m+mf}{0.5}\PY{p}{,} 
                                                     \PY{n}{min\PYZus{}lr}\PY{o}{=}\PY{l+m+mf}{0.00001}\PY{p}{)}
\end{Verbatim}


    \begin{Verbatim}[commandchars=\\\{\}]
{\color{incolor}In [{\color{incolor}36}]:} \PY{c+c1}{\PYZsh{} data augmentation}
         
         \PY{n}{datagen} \PY{o}{=} \PY{n}{ImageDataGenerator}\PY{p}{(}
                 \PY{n}{rotation\PYZus{}range}\PY{o}{=}\PY{l+m+mi}{30}\PY{p}{,}  \PY{c+c1}{\PYZsh{} randomly rotate images in the range (degrees, 0 to 180)}
                 \PY{n}{horizontal\PYZus{}flip}\PY{o}{=}\PY{k+kc}{True}\PY{p}{,}  \PY{c+c1}{\PYZsh{} randomly flip images}
                 \PY{n}{vertical\PYZus{}flip}\PY{o}{=}\PY{k+kc}{True}\PY{p}{)}  \PY{c+c1}{\PYZsh{} randomly flip images}
         
         \PY{n}{datagen}\PY{o}{.}\PY{n}{fit}\PY{p}{(}\PY{n}{x\PYZus{}train}\PY{p}{)}
\end{Verbatim}


    \begin{Verbatim}[commandchars=\\\{\}]
{\color{incolor}In [{\color{incolor}37}]:} \PY{c+c1}{\PYZsh{} train the model VGG RESULTS}
         \PY{c+c1}{\PYZsh{} epochs reduced and batch size increased due to slow runtime}
         \PY{n}{epochs} \PY{o}{=} \PY{l+m+mi}{20}
         \PY{n}{batch\PYZus{}size} \PY{o}{=} \PY{l+m+mi}{32}
         \PY{n}{history2} \PY{o}{=} \PY{n}{model}\PY{o}{.}\PY{n}{fit\PYZus{}generator}\PY{p}{(}\PY{n}{datagen}\PY{o}{.}\PY{n}{flow}\PY{p}{(}\PY{n}{x\PYZus{}train}\PY{p}{,}\PY{n}{y\PYZus{}train}\PY{p}{,} \PY{n}{batch\PYZus{}size}\PY{o}{=}\PY{n}{batch\PYZus{}size}\PY{p}{)}\PY{p}{,}
                                       \PY{n}{epochs} \PY{o}{=} \PY{n}{epochs}\PY{p}{,} \PY{n}{validation\PYZus{}data} \PY{o}{=} \PY{p}{(}\PY{n}{x\PYZus{}test}\PY{p}{,}\PY{n}{y\PYZus{}test}\PY{p}{)}\PY{p}{,}
                                       \PY{n}{verbose} \PY{o}{=} \PY{l+m+mi}{1}\PY{p}{,} \PY{n}{steps\PYZus{}per\PYZus{}epoch}\PY{o}{=}\PY{n}{x\PYZus{}train}\PY{o}{.}\PY{n}{shape}\PY{p}{[}\PY{l+m+mi}{0}\PY{p}{]} \PY{o}{/}\PY{o}{/} \PY{n}{batch\PYZus{}size}
                                       \PY{p}{,} \PY{n}{callbacks}\PY{o}{=}\PY{p}{[}\PY{n}{learning\PYZus{}rate\PYZus{}reduction}\PY{p}{]}\PY{p}{)}
\end{Verbatim}


    \begin{Verbatim}[commandchars=\\\{\}]
Epoch 1/20
234/234 [==============================] - 128s 546ms/step - loss: 5.3229 - acc: 0.6671 - val\_loss: 5.2847 - val\_acc: 0.6721
Epoch 2/20
234/234 [==============================] - 118s 506ms/step - loss: 5.3433 - acc: 0.6685 - val\_loss: 5.2847 - val\_acc: 0.6721
Epoch 3/20
234/234 [==============================] - 117s 501ms/step - loss: 5.3261 - acc: 0.6696 - val\_loss: 5.2847 - val\_acc: 0.6721
Epoch 4/20
234/234 [==============================] - 117s 502ms/step - loss: 5.3613 - acc: 0.6674 - val\_loss: 5.2847 - val\_acc: 0.6721

Epoch 00004: ReduceLROnPlateau reducing learning rate to 0.0005000000237487257.
Epoch 5/20
234/234 [==============================] - 117s 501ms/step - loss: 5.3179 - acc: 0.6701 - val\_loss: 5.2847 - val\_acc: 0.6721
Epoch 6/20
234/234 [==============================] - 118s 503ms/step - loss: 5.3665 - acc: 0.6670 - val\_loss: 5.2847 - val\_acc: 0.6721
Epoch 7/20
234/234 [==============================] - 117s 501ms/step - loss: 5.3214 - acc: 0.6698 - val\_loss: 5.2847 - val\_acc: 0.6721

Epoch 00007: ReduceLROnPlateau reducing learning rate to 0.0002500000118743628.
Epoch 8/20
234/234 [==============================] - 117s 501ms/step - loss: 5.3574 - acc: 0.6676 - val\_loss: 5.2847 - val\_acc: 0.6721
Epoch 9/20
234/234 [==============================] - 116s 498ms/step - loss: 5.3243 - acc: 0.6697 - val\_loss: 5.2847 - val\_acc: 0.6721
Epoch 10/20
234/234 [==============================] - 117s 498ms/step - loss: 5.3720 - acc: 0.6667 - val\_loss: 5.2847 - val\_acc: 0.6721

Epoch 00010: ReduceLROnPlateau reducing learning rate to 0.0001250000059371814.
Epoch 11/20
234/234 [==============================] - 117s 499ms/step - loss: 5.2930 - acc: 0.6716 - val\_loss: 5.2847 - val\_acc: 0.6721
Epoch 12/20
234/234 [==============================] - 117s 498ms/step - loss: 5.4225 - acc: 0.6636 - val\_loss: 5.2847 - val\_acc: 0.6721
Epoch 13/20
234/234 [==============================] - 116s 497ms/step - loss: 5.2912 - acc: 0.6717 - val\_loss: 5.2847 - val\_acc: 0.6721

Epoch 00013: ReduceLROnPlateau reducing learning rate to 6.25000029685907e-05.
Epoch 14/20
234/234 [==============================] - 117s 499ms/step - loss: 5.3511 - acc: 0.6680 - val\_loss: 5.2847 - val\_acc: 0.6721
Epoch 15/20
234/234 [==============================] - 117s 498ms/step - loss: 5.3273 - acc: 0.6695 - val\_loss: 5.2847 - val\_acc: 0.6721
Epoch 16/20
234/234 [==============================] - 117s 500ms/step - loss: 5.3481 - acc: 0.6682 - val\_loss: 5.2847 - val\_acc: 0.6721

Epoch 00016: ReduceLROnPlateau reducing learning rate to 3.125000148429535e-05.
Epoch 17/20
234/234 [==============================] - 117s 499ms/step - loss: 5.3257 - acc: 0.6696 - val\_loss: 5.2847 - val\_acc: 0.6721
Epoch 18/20
234/234 [==============================] - 116s 497ms/step - loss: 5.3175 - acc: 0.6701 - val\_loss: 5.2847 - val\_acc: 0.6721
Epoch 19/20
234/234 [==============================] - 116s 498ms/step - loss: 5.2955 - acc: 0.6715 - val\_loss: 5.2847 - val\_acc: 0.6721

Epoch 00019: ReduceLROnPlateau reducing learning rate to 1.5625000742147677e-05.
Epoch 20/20
234/234 [==============================] - 116s 497ms/step - loss: 5.4040 - acc: 0.6647 - val\_loss: 5.2847 - val\_acc: 0.6721

    \end{Verbatim}

    \begin{Verbatim}[commandchars=\\\{\}]
{\color{incolor}In [{\color{incolor}39}]:} \PY{c+c1}{\PYZsh{} thanks to https://www.kaggle.com/sid321axn/step\PYZhy{}wise\PYZhy{}approach\PYZhy{}cnn\PYZhy{}model\PYZhy{}77\PYZhy{}0344\PYZhy{}accuracy}
         \PY{c+c1}{\PYZsh{} function to plot model\PYZsq{}s validation loss and validation accuracy}
         \PY{k}{def} \PY{n+nf}{plot\PYZus{}model\PYZus{}history}\PY{p}{(}\PY{n}{model\PYZus{}history}\PY{p}{)}\PY{p}{:}
             \PY{n}{fig}\PY{p}{,} \PY{n}{axs} \PY{o}{=} \PY{n}{plt}\PY{o}{.}\PY{n}{subplots}\PY{p}{(}\PY{l+m+mi}{1}\PY{p}{,}\PY{l+m+mi}{2}\PY{p}{,}\PY{n}{figsize}\PY{o}{=}\PY{p}{(}\PY{l+m+mi}{15}\PY{p}{,}\PY{l+m+mi}{5}\PY{p}{)}\PY{p}{)}
             \PY{c+c1}{\PYZsh{} summarize history for accuracy}
             \PY{n}{axs}\PY{p}{[}\PY{l+m+mi}{0}\PY{p}{]}\PY{o}{.}\PY{n}{plot}\PY{p}{(}\PY{n+nb}{range}\PY{p}{(}\PY{l+m+mi}{1}\PY{p}{,}\PY{n+nb}{len}\PY{p}{(}\PY{n}{model\PYZus{}history}\PY{o}{.}\PY{n}{history}\PY{p}{[}\PY{l+s+s1}{\PYZsq{}}\PY{l+s+s1}{acc}\PY{l+s+s1}{\PYZsq{}}\PY{p}{]}\PY{p}{)}\PY{o}{+}\PY{l+m+mi}{1}\PY{p}{)}\PY{p}{,}\PY{n}{model\PYZus{}history}\PY{o}{.}\PY{n}{history}\PY{p}{[}\PY{l+s+s1}{\PYZsq{}}\PY{l+s+s1}{acc}\PY{l+s+s1}{\PYZsq{}}\PY{p}{]}\PY{p}{)}
             \PY{n}{axs}\PY{p}{[}\PY{l+m+mi}{0}\PY{p}{]}\PY{o}{.}\PY{n}{plot}\PY{p}{(}\PY{n+nb}{range}\PY{p}{(}\PY{l+m+mi}{1}\PY{p}{,}\PY{n+nb}{len}\PY{p}{(}\PY{n}{model\PYZus{}history}\PY{o}{.}\PY{n}{history}\PY{p}{[}\PY{l+s+s1}{\PYZsq{}}\PY{l+s+s1}{val\PYZus{}acc}\PY{l+s+s1}{\PYZsq{}}\PY{p}{]}\PY{p}{)}\PY{o}{+}\PY{l+m+mi}{1}\PY{p}{)}\PY{p}{,}\PY{n}{model\PYZus{}history}\PY{o}{.}\PY{n}{history}\PY{p}{[}\PY{l+s+s1}{\PYZsq{}}\PY{l+s+s1}{val\PYZus{}acc}\PY{l+s+s1}{\PYZsq{}}\PY{p}{]}\PY{p}{)}
             \PY{n}{axs}\PY{p}{[}\PY{l+m+mi}{0}\PY{p}{]}\PY{o}{.}\PY{n}{set\PYZus{}title}\PY{p}{(}\PY{l+s+s1}{\PYZsq{}}\PY{l+s+s1}{Model Accuracy}\PY{l+s+s1}{\PYZsq{}}\PY{p}{)}
             \PY{n}{axs}\PY{p}{[}\PY{l+m+mi}{0}\PY{p}{]}\PY{o}{.}\PY{n}{set\PYZus{}ylabel}\PY{p}{(}\PY{l+s+s1}{\PYZsq{}}\PY{l+s+s1}{Accuracy}\PY{l+s+s1}{\PYZsq{}}\PY{p}{)}
             \PY{n}{axs}\PY{p}{[}\PY{l+m+mi}{0}\PY{p}{]}\PY{o}{.}\PY{n}{set\PYZus{}xlabel}\PY{p}{(}\PY{l+s+s1}{\PYZsq{}}\PY{l+s+s1}{Epoch}\PY{l+s+s1}{\PYZsq{}}\PY{p}{)}
             \PY{n}{axs}\PY{p}{[}\PY{l+m+mi}{0}\PY{p}{]}\PY{o}{.}\PY{n}{set\PYZus{}xticks}\PY{p}{(}\PY{n}{np}\PY{o}{.}\PY{n}{arange}\PY{p}{(}\PY{l+m+mi}{1}\PY{p}{,}\PY{n+nb}{len}\PY{p}{(}\PY{n}{model\PYZus{}history}\PY{o}{.}\PY{n}{history}\PY{p}{[}\PY{l+s+s1}{\PYZsq{}}\PY{l+s+s1}{acc}\PY{l+s+s1}{\PYZsq{}}\PY{p}{]}\PY{p}{)}\PY{o}{+}\PY{l+m+mi}{1}\PY{p}{)}\PY{p}{,}\PY{n+nb}{len}\PY{p}{(}\PY{n}{model\PYZus{}history}\PY{o}{.}\PY{n}{history}\PY{p}{[}\PY{l+s+s1}{\PYZsq{}}\PY{l+s+s1}{acc}\PY{l+s+s1}{\PYZsq{}}\PY{p}{]}\PY{p}{)}\PY{o}{/}\PY{l+m+mi}{10}\PY{p}{)}
             \PY{n}{axs}\PY{p}{[}\PY{l+m+mi}{0}\PY{p}{]}\PY{o}{.}\PY{n}{legend}\PY{p}{(}\PY{p}{[}\PY{l+s+s1}{\PYZsq{}}\PY{l+s+s1}{train}\PY{l+s+s1}{\PYZsq{}}\PY{p}{,} \PY{l+s+s1}{\PYZsq{}}\PY{l+s+s1}{val}\PY{l+s+s1}{\PYZsq{}}\PY{p}{]}\PY{p}{,} \PY{n}{loc}\PY{o}{=}\PY{l+s+s1}{\PYZsq{}}\PY{l+s+s1}{best}\PY{l+s+s1}{\PYZsq{}}\PY{p}{)}
             \PY{c+c1}{\PYZsh{} summarize history for loss}
             \PY{n}{axs}\PY{p}{[}\PY{l+m+mi}{1}\PY{p}{]}\PY{o}{.}\PY{n}{plot}\PY{p}{(}\PY{n+nb}{range}\PY{p}{(}\PY{l+m+mi}{1}\PY{p}{,}\PY{n+nb}{len}\PY{p}{(}\PY{n}{model\PYZus{}history}\PY{o}{.}\PY{n}{history}\PY{p}{[}\PY{l+s+s1}{\PYZsq{}}\PY{l+s+s1}{loss}\PY{l+s+s1}{\PYZsq{}}\PY{p}{]}\PY{p}{)}\PY{o}{+}\PY{l+m+mi}{1}\PY{p}{)}\PY{p}{,}\PY{n}{model\PYZus{}history}\PY{o}{.}\PY{n}{history}\PY{p}{[}\PY{l+s+s1}{\PYZsq{}}\PY{l+s+s1}{loss}\PY{l+s+s1}{\PYZsq{}}\PY{p}{]}\PY{p}{)}
             \PY{n}{axs}\PY{p}{[}\PY{l+m+mi}{1}\PY{p}{]}\PY{o}{.}\PY{n}{plot}\PY{p}{(}\PY{n+nb}{range}\PY{p}{(}\PY{l+m+mi}{1}\PY{p}{,}\PY{n+nb}{len}\PY{p}{(}\PY{n}{model\PYZus{}history}\PY{o}{.}\PY{n}{history}\PY{p}{[}\PY{l+s+s1}{\PYZsq{}}\PY{l+s+s1}{val\PYZus{}loss}\PY{l+s+s1}{\PYZsq{}}\PY{p}{]}\PY{p}{)}\PY{o}{+}\PY{l+m+mi}{1}\PY{p}{)}\PY{p}{,}\PY{n}{model\PYZus{}history}\PY{o}{.}\PY{n}{history}\PY{p}{[}\PY{l+s+s1}{\PYZsq{}}\PY{l+s+s1}{val\PYZus{}loss}\PY{l+s+s1}{\PYZsq{}}\PY{p}{]}\PY{p}{)}
             \PY{n}{axs}\PY{p}{[}\PY{l+m+mi}{1}\PY{p}{]}\PY{o}{.}\PY{n}{set\PYZus{}title}\PY{p}{(}\PY{l+s+s1}{\PYZsq{}}\PY{l+s+s1}{Model Loss}\PY{l+s+s1}{\PYZsq{}}\PY{p}{)}
             \PY{n}{axs}\PY{p}{[}\PY{l+m+mi}{1}\PY{p}{]}\PY{o}{.}\PY{n}{set\PYZus{}ylabel}\PY{p}{(}\PY{l+s+s1}{\PYZsq{}}\PY{l+s+s1}{Loss}\PY{l+s+s1}{\PYZsq{}}\PY{p}{)}
             \PY{n}{axs}\PY{p}{[}\PY{l+m+mi}{1}\PY{p}{]}\PY{o}{.}\PY{n}{set\PYZus{}xlabel}\PY{p}{(}\PY{l+s+s1}{\PYZsq{}}\PY{l+s+s1}{Epoch}\PY{l+s+s1}{\PYZsq{}}\PY{p}{)}
             \PY{n}{axs}\PY{p}{[}\PY{l+m+mi}{1}\PY{p}{]}\PY{o}{.}\PY{n}{set\PYZus{}xticks}\PY{p}{(}\PY{n}{np}\PY{o}{.}\PY{n}{arange}\PY{p}{(}\PY{l+m+mi}{1}\PY{p}{,}\PY{n+nb}{len}\PY{p}{(}\PY{n}{model\PYZus{}history}\PY{o}{.}\PY{n}{history}\PY{p}{[}\PY{l+s+s1}{\PYZsq{}}\PY{l+s+s1}{loss}\PY{l+s+s1}{\PYZsq{}}\PY{p}{]}\PY{p}{)}\PY{o}{+}\PY{l+m+mi}{1}\PY{p}{)}\PY{p}{,}\PY{n+nb}{len}\PY{p}{(}\PY{n}{model\PYZus{}history}\PY{o}{.}\PY{n}{history}\PY{p}{[}\PY{l+s+s1}{\PYZsq{}}\PY{l+s+s1}{loss}\PY{l+s+s1}{\PYZsq{}}\PY{p}{]}\PY{p}{)}\PY{o}{/}\PY{l+m+mi}{10}\PY{p}{)}
             \PY{n}{axs}\PY{p}{[}\PY{l+m+mi}{1}\PY{p}{]}\PY{o}{.}\PY{n}{legend}\PY{p}{(}\PY{p}{[}\PY{l+s+s1}{\PYZsq{}}\PY{l+s+s1}{train}\PY{l+s+s1}{\PYZsq{}}\PY{p}{,} \PY{l+s+s1}{\PYZsq{}}\PY{l+s+s1}{val}\PY{l+s+s1}{\PYZsq{}}\PY{p}{]}\PY{p}{,} \PY{n}{loc}\PY{o}{=}\PY{l+s+s1}{\PYZsq{}}\PY{l+s+s1}{best}\PY{l+s+s1}{\PYZsq{}}\PY{p}{)}
             \PY{n}{plt}\PY{o}{.}\PY{n}{show}\PY{p}{(}\PY{p}{)}
\end{Verbatim}


    \begin{Verbatim}[commandchars=\\\{\}]
{\color{incolor}In [{\color{incolor}41}]:} \PY{n}{plot\PYZus{}model\PYZus{}history}\PY{p}{(}\PY{n}{history2}\PY{p}{)}
\end{Verbatim}


    \begin{center}
    \adjustimage{max size={0.9\linewidth}{0.9\paperheight}}{output_86_0.png}
    \end{center}
    { \hspace*{\fill} \\}
    
    Even though VGG was not successful in working with the test data, we are
still able to perform an analysis on the training data. Despite
providing data annealing, the variation is high throughout every epoch
suggesting that the learning rate was still too high, and data annealing
needed to be applied more aggressively. On the other hand, you can
actually see that the variance is only for a very small range and in
reality the learning rate may not have been high enough to begin with.
Even though the accuracy lies at a respectable 67.21\%, as the
validation accuracy and loss are flat it is likely that the model does
not work as intended and the model is not actually learning anything
about the images over time. This could potentially be explained to a
very low training rate, and is something to keep in consideration for
future implementation of VGG for this dataset. Due to VGG being quite
demanding in terms of resources and taking large amounts of time each
time the model is trained it was not possible to tune the parameters in
many ways and try to `fix' the model. This is something that can be
explored in future projects.

    \begin{Verbatim}[commandchars=\\\{\}]
{\color{incolor}In [{\color{incolor}42}]:} \PY{c+c1}{\PYZsh{} thanks to https://www.kaggle.com/sid321axn/step\PYZhy{}wise\PYZhy{}approach\PYZhy{}cnn\PYZhy{}model\PYZhy{}77\PYZhy{}0344\PYZhy{}accuracy}
         
         \PY{n}{y\PYZus{}pred} \PY{o}{=} \PY{n}{model}\PY{o}{.}\PY{n}{predict}\PY{p}{(}\PY{n}{x\PYZus{}test}\PY{p}{)}
         \PY{n}{y\PYZus{}pred} \PY{o}{=} \PY{n}{np}\PY{o}{.}\PY{n}{argmax}\PY{p}{(}\PY{n}{y\PYZus{}pred}\PY{p}{,} \PY{n}{axis} \PY{o}{=} \PY{l+m+mi}{1}\PY{p}{)}
         \PY{n}{y\PYZus{}true} \PY{o}{=} \PY{n}{np}\PY{o}{.}\PY{n}{argmax}\PY{p}{(}\PY{n}{y\PYZus{}test}\PY{p}{,}\PY{n}{axis} \PY{o}{=} \PY{l+m+mi}{1}\PY{p}{)} 
         
         \PY{n}{confusion} \PY{o}{=} \PY{n}{confusion\PYZus{}matrix}\PY{p}{(}\PY{n}{y\PYZus{}pred}\PY{p}{,} \PY{n}{y\PYZus{}true}\PY{p}{)}
         
         \PY{c+c1}{\PYZsh{} Function to plot confusion matrix    }
         \PY{k}{def} \PY{n+nf}{plot\PYZus{}confusion\PYZus{}matrix}\PY{p}{(}\PY{n}{cm}\PY{p}{,} \PY{n}{classes}\PY{p}{,}
                                   \PY{n}{normalize}\PY{o}{=}\PY{k+kc}{False}\PY{p}{,}
                                   \PY{n}{title}\PY{o}{=}\PY{l+s+s1}{\PYZsq{}}\PY{l+s+s1}{Confusion matrix}\PY{l+s+s1}{\PYZsq{}}\PY{p}{,}
                                   \PY{n}{cmap}\PY{o}{=}\PY{n}{plt}\PY{o}{.}\PY{n}{cm}\PY{o}{.}\PY{n}{Blues}\PY{p}{)}\PY{p}{:}
             \PY{l+s+sd}{\PYZdq{}\PYZdq{}\PYZdq{}}
         \PY{l+s+sd}{    This function prints and plots the confusion matrix.}
         \PY{l+s+sd}{    Normalization can be applied by setting `normalize=True`.}
         \PY{l+s+sd}{    \PYZdq{}\PYZdq{}\PYZdq{}}
             \PY{n}{plt}\PY{o}{.}\PY{n}{imshow}\PY{p}{(}\PY{n}{cm}\PY{p}{,} \PY{n}{interpolation}\PY{o}{=}\PY{l+s+s1}{\PYZsq{}}\PY{l+s+s1}{nearest}\PY{l+s+s1}{\PYZsq{}}\PY{p}{,} \PY{n}{cmap}\PY{o}{=}\PY{n}{cmap}\PY{p}{)}
             \PY{n}{plt}\PY{o}{.}\PY{n}{title}\PY{p}{(}\PY{n}{title}\PY{p}{)}
             \PY{n}{plt}\PY{o}{.}\PY{n}{colorbar}\PY{p}{(}\PY{p}{)}
             \PY{n}{tick\PYZus{}marks} \PY{o}{=} \PY{n}{np}\PY{o}{.}\PY{n}{arange}\PY{p}{(}\PY{n+nb}{len}\PY{p}{(}\PY{n}{classes}\PY{p}{)}\PY{p}{)}
             \PY{n}{plt}\PY{o}{.}\PY{n}{xticks}\PY{p}{(}\PY{n}{tick\PYZus{}marks}\PY{p}{,} \PY{n}{classes}\PY{p}{,} \PY{n}{rotation}\PY{o}{=}\PY{l+m+mi}{45}\PY{p}{)}
             \PY{n}{plt}\PY{o}{.}\PY{n}{yticks}\PY{p}{(}\PY{n}{tick\PYZus{}marks}\PY{p}{,} \PY{n}{classes}\PY{p}{)}
         
             \PY{k}{if} \PY{n}{normalize}\PY{p}{:}
                 \PY{n}{cm} \PY{o}{=} \PY{n}{cm}\PY{o}{.}\PY{n}{astype}\PY{p}{(}\PY{l+s+s1}{\PYZsq{}}\PY{l+s+s1}{float}\PY{l+s+s1}{\PYZsq{}}\PY{p}{)} \PY{o}{/} \PY{n}{cm}\PY{o}{.}\PY{n}{sum}\PY{p}{(}\PY{n}{axis}\PY{o}{=}\PY{l+m+mi}{1}\PY{p}{)}\PY{p}{[}\PY{p}{:}\PY{p}{,} \PY{n}{np}\PY{o}{.}\PY{n}{newaxis}\PY{p}{]}
         
             \PY{n}{thresh} \PY{o}{=} \PY{n}{cm}\PY{o}{.}\PY{n}{max}\PY{p}{(}\PY{p}{)} \PY{o}{/} \PY{l+m+mf}{2.}
             \PY{k}{for} \PY{n}{i}\PY{p}{,} \PY{n}{j} \PY{o+ow}{in} \PY{n}{itertools}\PY{o}{.}\PY{n}{product}\PY{p}{(}\PY{n+nb}{range}\PY{p}{(}\PY{n}{cm}\PY{o}{.}\PY{n}{shape}\PY{p}{[}\PY{l+m+mi}{0}\PY{p}{]}\PY{p}{)}\PY{p}{,} \PY{n+nb}{range}\PY{p}{(}\PY{n}{cm}\PY{o}{.}\PY{n}{shape}\PY{p}{[}\PY{l+m+mi}{1}\PY{p}{]}\PY{p}{)}\PY{p}{)}\PY{p}{:}
                 \PY{n}{plt}\PY{o}{.}\PY{n}{text}\PY{p}{(}\PY{n}{j}\PY{p}{,} \PY{n}{i}\PY{p}{,} \PY{n}{cm}\PY{p}{[}\PY{n}{i}\PY{p}{,} \PY{n}{j}\PY{p}{]}\PY{p}{,}
                          \PY{n}{horizontalalignment}\PY{o}{=}\PY{l+s+s2}{\PYZdq{}}\PY{l+s+s2}{center}\PY{l+s+s2}{\PYZdq{}}\PY{p}{,}
                          \PY{n}{color}\PY{o}{=}\PY{l+s+s2}{\PYZdq{}}\PY{l+s+s2}{white}\PY{l+s+s2}{\PYZdq{}} \PY{k}{if} \PY{n}{cm}\PY{p}{[}\PY{n}{i}\PY{p}{,} \PY{n}{j}\PY{p}{]} \PY{o}{\PYZgt{}} \PY{n}{thresh} \PY{k}{else} \PY{l+s+s2}{\PYZdq{}}\PY{l+s+s2}{black}\PY{l+s+s2}{\PYZdq{}}\PY{p}{)}
         
             \PY{n}{plt}\PY{o}{.}\PY{n}{tight\PYZus{}layout}\PY{p}{(}\PY{p}{)}
             \PY{n}{plt}\PY{o}{.}\PY{n}{ylabel}\PY{p}{(}\PY{l+s+s1}{\PYZsq{}}\PY{l+s+s1}{True label}\PY{l+s+s1}{\PYZsq{}}\PY{p}{)}
             \PY{n}{plt}\PY{o}{.}\PY{n}{xlabel}\PY{p}{(}\PY{l+s+s1}{\PYZsq{}}\PY{l+s+s1}{Predicted label}\PY{l+s+s1}{\PYZsq{}}\PY{p}{)}
         
         \PY{n}{plot\PYZus{}confusion\PYZus{}matrix}\PY{p}{(}\PY{n}{confusion}\PY{p}{,} \PY{n}{classes} \PY{o}{=} \PY{n+nb}{range}\PY{p}{(}\PY{l+m+mi}{7}\PY{p}{)}\PY{p}{)}
\end{Verbatim}


    \begin{Verbatim}[commandchars=\\\{\}]

        ---------------------------------------------------------------------------

        NameError                                 Traceback (most recent call last)

        <ipython-input-42-e8c4038ddc62> in <module>
         34     plt.xlabel('Predicted label')
         35 
    ---> 36 plot\_confusion\_matrix(confusion, classes = range(7))
    

        <ipython-input-42-e8c4038ddc62> in plot\_confusion\_matrix(cm, classes, normalize, title, cmap)
         25 
         26     thresh = cm.max() / 2.
    ---> 27     for i, j in itertools.product(range(cm.shape[0]), range(cm.shape[1])):
         28         plt.text(j, i, cm[i, j],
         29                  horizontalalignment="center",
    

        NameError: name 'itertools' is not defined

    \end{Verbatim}

    \begin{center}
    \adjustimage{max size={0.9\linewidth}{0.9\paperheight}}{output_88_1.png}
    \end{center}
    { \hspace*{\fill} \\}
    
    The confusion matrix for VGG is very similar to that of the custom built
CNN, and also suggests that the model is overfitted, potentially giving
an explanation for the results that were gathered.

    Interpretation

While it is likely that the results are not accurate, a comparison will
still be made with the assumption that the results are `correct'. Only
the test data is compared, as training data results are not very
relevant for comparisons between models. While the custom built CNN ends
with an accuracy of 66.49\%, VGG ends with the slightly higher accuracy
of 67.21\%. Similarly, the custom built CNN ends with a loss of 5.3748,
whereas VGG ends with a loss of 5.2847. Taking both of these into
account, VGG has the highest accuracy and lowest loss by a small margin,
making it slightly better than the custom built model. It is also worth
noting that it was significantly slower to train VGG than the custom
built CNN, and so if the dataset is large enough it may be worth the
tradeoff for a small amount of optimisation in loss and accuracy to get
much faster run times. However, to reiterate, it is likely that the
models are not working as intended and these results should be treated
cautiously.

    How Successful was my Project?

While a conclusion was met that satisfied the initial goal, the project
may be deemed as unsuccesful as the models are likely not functioning as
intended. However, I would argue that the project has seen success in
many ways. For example, the groundwork has been layed out for anyone to
pick up the project, add in a model of their choosing that they would
like to compare, fine-tune the hyperparameters and use the tools
provided such as the graphs for the models and confusion matrices for a
comparison between any model. Another example would be that I have not
yet discovered an attempt for an implementation of VGG on this dataset,
and therefore the groundwork has also been layed out for the projects
code to be tuned in such a way that VGG may receive more desireable
results.

If using `The Logical Framework Method for Defining Project Success'
(Baccarini, 1999), the project can be deemed a failure in terms of
`project management success' as the original `deliverable' was not
achieved as desired. However, the project success can be deemed as a
success as groundwork has been provided for future work in this area and
although it was not the original deliverable, a successful output has
been produce. The paper also makes a very good point about success not
being `black or white' and it may be more suitable to deem this project
as a partial success.

In the future, I believe that a large scale comparison is needed, as if
the `most efficient' model is found, it could potentially be used in
place of diagnosing by eye providing automated and fast diagnoses for
dermatoscopic images.

    Reflection

This project also introduced me to many new technologies, which has been
a pleasure. As this was my first large project with python, it was nice
to learn a new programming language and the strengths and weaknesses
that it comes with. Python was a very successful language for machine
learning as it is widely supported and relatively simple to pick up, key
for the short time scale of the project. As a Data Scientist, I believe
that both Python and R will prove to be useful tools to use in future
projects as they excel in many tasks that are required of data science
projects. Anaconda was a great way to produce environments for the
project and easily download and manage the packages required for the
code. Jupyter Notebooks was a great tool for producing markdown
documents such as this one, and provides numerous advantages and
disadvantages over R, such as code being much easier to manage in my
opinion, but also a lot more tedious to use with the natural
implementation of cells for programming. Cookie cutter lacks the
automation that Project Template provides, although it adds a lot more
customisation in that there are a huge amount of premade directories
online, which allowed me to find the data science cookie cutter build
which worked well for this project.

KDD was much more successful for this project than the cloud computing
project, as it utilised all of the steps (cloud computing did not make
use of the three data mining steps). However, that is not to say that it
did not come without issues. As previously mentioned, the linearity of
KDD can result in wasted time, or if followed precisely with no going
back between steps can actually force a project to stop before starting
another `cycle' after the interpretation phase.

Ironically, the most straightforward part was likely implementing the
models themselves (Although not to much success), as there are many
existing frameworks that you can base yours off of. The most difficult
part of the project for me was editing the models and data in a way such
that accuracy and loss are optimised. When given a large dataset such as
this one computation times can be quite slow which does not give many
attempts at fine tuning the models.

For future projects, it is always worth following a methodology as it
provides a natural work flow and clear objectives to achieve your goal,
although I would like to explore a new methodology in my next project
such as AGILE or SCRUM. It may also be worth considering my
strengths/weaknesses for future projects, as while I attempted to tackle
a quite difficult dataset in domains I have never worked with before
(Medical and images for machine learning) which definitely improved my
skills and abilities, it also led to the original goal not being met.
Therefore, rather than taking on a whole new area at once, it may be
wise to approach a project with more caution.

    References

Tschandl, P., Rosendahl, C. and Kittler, H. (2018). The HAM10000
dataset, a large collection of multi-source dermatoscopic images of
common pigmented skin lesions. Scientific Data, 5, p.180161.

Howard, J. (2019). fast.ai · Making neural nets uncool again.
{[}online{]} Fast.ai. Available at: https://www.fast.ai/ {[}Accessed 22
Jan.~2019{]}.

Keskar, N.S., Mudigere, D., Nocedal, J., Smelyanskiy, M. and Tang,
P.T.P., 2016. On large-batch training for deep learning: Generalization
gap and sharp minima. arXiv preprint arXiv:1609.04836.

Caruana, R. and Niculescu-Mizil, A. (2006). An empirical comparison of
supervised learning algorithms. Proceedings of the

23rd international conference on Machine learning - ICML '06.

Simonyan, K. and Zisserman, A., 2014. Very deep convolutional networks
for large-scale image recognition. arXiv preprint arXiv:1409.1556.

Medium. (2019). The best explanation of Convolutional Neural Networks on
the Internet!. {[}online{]} Available at:
https://medium.com/technologymadeeasy/the-best-explanation-of-convolutional-neural-networks-on-the-internet-fbb8b1ad5df8
{[}Accessed 23 Jan.~2019{]}.

Medium. (2019). Cookiecutter Data Science --- Organize your Projects ---
Atom and Jupyter. {[}online{]} Available at:
https://medium.com/@rrfd/cookiecutter-data-science-organize-your-projects-atom-and-jupyter-2be7862f487e
{[}Accessed 7 Jan. 2019{]}.

Deshpande, A. (2019). A Beginner's Guide To Understanding Convolutional
Neural Networks. {[}online{]} Adeshpande3.github.io. Available at:
https://adeshpande3.github.io/A-Beginner\%27s-Guide-To-Understanding-Convolutional-Neural-Networks/
{[}Accessed 24 Jan.~2019{]}.

Baccarini, D. (1999). The Logical Framework Method for Defining Project
Success. Project Management Journal, 30(4), pp.25-32.


    % Add a bibliography block to the postdoc
    
    
    
    \end{document}
